\documentclass[a4paper,12pt]{article}

\usepackage[utf8]{inputenc}
\usepackage[spanish]{babel}
\usepackage{csquotes}
\usepackage{graphicx}
\usepackage{hyperref}

\hypersetup{
    colorlinks=true,
    linkcolor=blue,
    filecolor=magenta,      
    urlcolor=cyan,
}

% Para tablas complejas
\usepackage{multicol,multirow}
\usepackage{array,longtable}
% automatic math mode, centered
\newcolumntype{C}{>{\(}c<{\)}}
\newcolumntype{R}{>{\(}r<{\)}}
\newcolumntype{L}{>{\(}l<{\)}}

\usepackage{amsmath,amsfonts,amssymb,amsthm}
\allowdisplaybreaks % Para que el align no te pida %
\usepackage[bb=boondox]{mathalfa}

\begin{document}

\title{(Pseudo) Bit\'acora de la PFI}
\author{Apellido, Nombre}
\date{\today}
\maketitle

\begin{abstract}
    La idea de este documento es tener un espacio donde documentar algunos resultados valiosos, cambios importantes, decisiones tomadas y el por qu\'e de las mismas.
\end{abstract}

\tableofcontents
\newpage

\input{history/considerations} \newpage
\input{history/01} \newpage
\input{history/02} \newpage

\end{document}
