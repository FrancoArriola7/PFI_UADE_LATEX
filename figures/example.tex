%h (here), le decimos que ponga la imagen m\'as o menos aqu\'{\i}
%t (top), preferiblemente en la parte superior de la p\'agina
%b (bottom), preferiblemente en la parte inferior de la p\'agina
%p (page), que junte los objetos flotantes en una p\'agina
%! que ignore sus reglas internas de posicionamiento
%H que ponga la imagen justo aqu\'{\i}, similar a h!
\begin{figure}[t]
	\centering
	\begin{framed}
        Contenido de la figura
    \end{framed}
    \caption{Ejemplo de Figura}
	\label{fig:example}
\end{figure}