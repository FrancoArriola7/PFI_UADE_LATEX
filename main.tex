% oneside para una impresión simple.
% titlepage define que vamos a tener una portada.
% openany para que los capítulos empiecen en cualquier página.
\documentclass[a4paper,12pt,oneside,titlepage,openany]{book}

% --- Seccionado temprano
\usepackage{titlesec}

% --- Encoding / fuentes
\usepackage{cmap}
\usepackage[utf8]{inputenc}
\usepackage[T1]{fontenc}
\usepackage{mathptmx} % Times (como en el template nuevo)

% --- Bibliografía (biber, estilo del profe)
\usepackage[backend=biber,style=iso-authoryear,language=spanish]{biblatex}
\DeclareUnicodeCharacter{202F}{~}
\addbibresource{biblio.bib}
\setcounter{tocdepth}{3}
\setcounter{secnumdepth}{3}
\DefineBibliographyStrings{spanish}{
  online   = {En línea},
  urlfrom  = {Disponible en:},
  andothers = {\mkbibemph{et\addabbrvspace al\adddot}}
}
\DeclareFieldFormat{urldate}{\mkbibbrackets{consultado\space#1}}

% --- Colores / flotantes / microtipografía
\usepackage[usenames]{color}
\usepackage{framed}
\usepackage[section,above,below]{placeins}
\usepackage{microtype}
\usepackage{anyfontsize}

% --- Listas y marcadores personalizados
\usepackage{enumitem}
\newlist{Properties}{enumerate}{2}
\setlist[Properties]{label=Propiedad \arabic*.,itemindent=*}
\renewcommand{\labelitemii}{\textasteriskcentered}
\renewcommand{\labelitemiii}{-}

% --- Encabezados/pies y portada PDF
\usepackage{lastpage}
\usepackage{titleps}
\usepackage{textcase}
\usepackage{pdfpages}

% Encabezado con tu logo/título/autores (de tu main.tex original)
\newpagestyle{ruled}{
  \sethead{\includegraphics[width=3.7cm]{./images/UADE_LARGE}}
          {\parbox[b]{0.43\textwidth}{\centering AIVA Sistema de Predicción de Demanda con IA para productos perecederos}}
          {\shortstack{Arriola, Franco \\ Janse, Bautista}}
  \headrule{}
  \setfoot[][][Página \thepage\ de~\pageref*{LastPage}]{}{}{Página \thepage\ de~\pageref*{LastPage}}
  \footrule{}
}
\pagestyle{ruled}

% Para que la primera página de cada capítulo use el mismo estilo
\makeatletter
\let\ps@plain\ps@ruled
\makeatother

% --- Márgenes (como usaban)
\usepackage{vmargin}
\setmarginsrb{2.7cm}{2cm}{2.3cm}{1.5cm}{0cm}{2cm}{0cm}{1cm}

% --- Numeración de tablas
\renewcommand{\thetable}{\thechapter.\Roman{table}}

% --- Espaciado de flotantes
\setlength{\textfloatsep}{5pt}

% --- Notas/tareas (tus macros)
\usepackage{todonotes}
\setlength{\marginparwidth}{2cm}
\newcommand{\Nico}[1]{\todo[color=green!30,inline]{\textbf{Nico:} #1}}
\newcommand{\Franco}[1]{\todo[color=yellow!30,inline]{\textbf{Franco:} #1}}
\newcommand{\Bauti}[1]{\todo[color=blue!30,inline]{\textbf{Bauti:} #1}}

% --- Bibliografía: espaciado en lista
\usepackage{etoolbox}
\patchcmd{\thebibliography}{\settowidth}{\setlength{\itemsep}{6pt}\settowidth}{}{}
\apptocmd{\thebibliography}{\small}{}{}

% --- ToC/LoF/LoT (una sola vez)
\usepackage{tocloft}
\cftsetindents{table}{0em}{4em}
\renewcommand{\cfttoctitlefont}{\bfseries\fontsize{14pt}{16pt}\selectfont}
\setlength{\cftbeforetoctitleskip}{12pt}
\setlength{\cftaftertoctitleskip}{6pt}
\renewcommand{\cftloftitlefont}{\bfseries\fontsize{14pt}{16pt}\selectfont}
\renewcommand{\cftlottitlefont}{\bfseries\fontsize{14pt}{16pt}\selectfont}
\setlength{\cftbeforeloftitleskip}{12pt}
\setlength{\cftafterloftitleskip}{6pt}
\setlength{\cftbeforelottitleskip}{12pt}
\setlength{\cftafterlottitleskip}{6pt}

% --- Idioma / captions
\usepackage[spanish,es-nodecimaldot]{babel}
\usepackage{csquotes}
\addto\captionsspanish{
  \renewcommand{\contentsname}{Índice}
  \renewcommand{\listfigurename}{Lista de Figuras}
  \renewcommand{\listtablename}{Lista de Tablas}
  \renewcommand{\tablename}{TABLA}
}

% --- Párrafo
\usepackage{indentfirst}
\setlength{\parindent}{36pt}

% --- Tablas complejas (de tu proyecto)
\usepackage{tabularx}
\usepackage{booktabs}
\usepackage{multicol,multirow}
\usepackage{array,longtable}

% --- Matemática
\usepackage{amsmath,amsfonts,amssymb,amsthm}
\allowdisplaybreaks{}
\usepackage{cancel}

% --- Entornos propios (sin espacios en el nombre)
\newenvironment{EstadoDelArte}{\small\begin{center}\bfseries EstadoDelArte\end{center}}{}
\newenvironment{EnfoqueMetodologico}{\small\begin{center}\bfseries Enfoque metodológico\end{center}}{}
\newenvironment{Cronograma}{\small\begin{center}\bfseries Cronograma\end{center}}{}

% --- Símbolos de tu repo
\input{config/symbols}

% --- Subrayados / estilo
\usepackage[normalem]{ulem}

% --- Interlineado 1.5 del template nuevo
\usepackage{setspace}
\onehalfspacing

% --- Títulos (UADE: todo 14 pt)
\titleformat{\chapter}{\bfseries\fontsize{14pt}{14pt}\selectfont}{\thechapter}{1em}{}
\titlespacing*{\chapter}{0pt}{12pt}{6pt}
\titleformat{\section}{\bfseries\fontsize{14pt}{14pt}\selectfont}{\thesection}{1em}{}
\titlespacing*{\section}{0pt}{12pt}{6pt}
\titleformat{\subsection}{\bfseries\fontsize{14pt}{14pt}\selectfont}{\thesubsection}{1em}{}
\titlespacing*{\subsection}{0pt}{12pt}{6pt}
\titleformat{\subsubsection}{\bfseries\fontsize{14pt}{14pt}\selectfont}{\thesubsubsection}{1em}{}
\titlespacing*{\subsubsection}{0pt}{12pt}{6pt}

% --- Bibliografía en el índice automáticamente
\usepackage[nottoc]{tocbibind}

% --- Hipervínculos: para entrega UADE sin color
\usepackage[hidelinks]{hyperref}

\begin{document}

% Portada en PDF del template nuevo (comenta si no aplica)
\clearpage
\thispagestyle{empty}
\includepdf[
  pages=1,
  scale=1.01,
  offset=25mm -25mm,
  pagecommand={\thispagestyle{empty}},
  noautoscale
]{cover/f0b8227c-9dcd-4a45-8570-f0ce3274233d.pdf}
\clearpage

% --- Contenido (tu estructura original)
\begin{titlepage} % \maketitle

	\centering
	%\fontsize{<size>}{<bskip>}
	{\textbf{\fontsize{16}{17}\selectfont TÍTULO DE LA PFI} \par}
	\vspace{1cm}
	{\textbf{\fontsize{16}{17}\selectfont Apellido, Nombres - LU XXXXXXX} \par}
	{\textbf{\fontsize{16}{17}\selectfont Apellido, Nombres - LU XXXXXXX} \par}
	\vspace{1.5cm}
	{\fontsize{16}{17}\selectfont Ingenier\'{\i}a en Inform\'atica \par}
	\vspace{1cm}
	{\textbf{\fontsize{14}{14}\selectfont Tutor:} \par}
	{\fontsize{14}{14}\selectfont Monzón, Nicolás Alberto
		\\ (UADE) Universidad Argentina de la Empresa Lima 757, Cdad. Autónoma de Buenos Aires, Argetina.
		\\ (UdelaR) Universidad de la República, Av. 18 de Julio 1968, 11200 Montevideo, Departamento de Montevideo, Uruguay.
		\par}
	\vspace{1cm}
	\vfill
	\today
	\vfill
	\includegraphics[width=0.30\textwidth]{./images/UADE}\par \vspace{1cm}
	{\textbf{\fontsize{14}{14}\selectfont UNIVERSIDAD ARGENTINA DE LA EMPRESA} \par}
	{\fontsize{14}{14}\selectfont FACULTAD DE INGENIER\'IA Y CIENCIAS EXACTAS \par}

\end{titlepage}
\begin{Agradecimientos}

Completar.

\end{Agradecimientos}
\newpage
\begin{Resumen} % \chapter*{Resumen}

    \indent El presente Proyecto Final de Ingeniería propone el desarrollo de un Sistema de análisis y predicción con inteligencia artificial para tiendas de productos perecederos. El proyecto tiene por objetivo optimizar la producción diaria de pequeños y medianos comercios de Buenos Aires en 2025 mediante la predicción de la demanda, reduciendo el desperdicio de alimentos y mejorando la rentabilidad.\\

    \indent La plataforma integrará modelos de series temporales e IA para generar pronósticos por artículo, procesará registros de venta extraídos de imágenes y ofrecerá un motor analítico que, a través de un dashboard y un chatbot conversacional, brindará indicadores clave y recomendaciones accionables al usuario. El MVP incluirá: módulo de predicción, visualización de KPIs, carga de imágenes, alertas de sobreproducción o faltantes y comparaciones semana a semana.\\

    \indent Entre los beneficios esperados destacan: reducción de mermas, incremento del margen operativo, profesionalización de la gestión y optimización de la compra de insumos, buscando dotar a los comercios perecederos de una herramienta innovadora que alinee su producción con la demanda real, contribuya a la sostenibilidad alimentaria y fortalezca su competitividad en un entorno económico desafiante.

\end{Resumen}

\newpage
\begin{Abstract}

    \indent This Final Engineering Project proposes the development of an AI-driven analysis and prediction system for stores that sell perishable goods. Conceived as a web-based software product, the project seeks to optimize the daily production of small and medium-sized retailers in Buenos Aires in 2025 by forecasting demand, thereby reducing food waste and improving profitability.\\

    \indent The platform will combine time-series forecasting models and artificial intelligence to generate item-level predictions, process sales records extracted from images, and provide an analytics engine that—through a dashboard and a conversational chatbot—delivers key indicators and actionable recommendations to users. The MVP will comprise a prediction module, KPI visualizations, image ingestion, alerts for overproduction or stock-outs, and week-over-week comparisons.\\
    
    \indent Expected benefits include lower shrinkage, higher operating margins, more professional management, and optimized procurement of raw materials. Ultimately, the system aims to furnish perishable-goods retailers with an innovative tool that aligns production with real demand, supports food sustainability, and bolsters competitiveness in a challenging economic environment.

\end{Abstract}
\newpage

\tableofcontents

\chapter{Introducción}

El desperdicio de alimentos constituye uno de los desafíos socio-ambientales más apremiantes de la actualidad. Estudios sobre retail alimentario local muestran, además, que por cada 100 pesos vendidos en categorías de frescos y almacén se pierden más de 3 pesos en mermas y obsolescencia, afectando márgenes ya de por sí estrechos \parencite{weteam2021}. Además, en la Argentina se pierde y desperdicia anualmente el 12,5\% de la producción agroalimentaria, equivalente a 16 millones de toneladas, de las cuales solo el 1,2\% corresponde a desperdicio en la etapa de comercialización, mientras que el resto se debe a pérdidas a lo largo de la cadena \parencite{tiscornia2022}. Esta situación implica un derroche de recursos naturales y económicos que afecta de manera directa la rentabilidad de los comercios que operan con productos frescos y perecederos.

El segmento de panaderías y pastelerías ilustra con crudeza esta problemática: en 2024 cerraron más de 400 locales, reflejando la dificultad de muchos pequeños y medianos negocios para ajustar su producción a una demanda cada vez más volátil (Pinto, 2024). 

Frente a este escenario, el presente Proyecto Final de Ingeniería propone el desarrollo de un sistema de análisis y predicción con inteligencia artificial para tiendas de productos perecederos, concebido como una plataforma web de fácil adopción. El sistema integrará modelos de series temporales y aprendizaje automático para pronosticar la demanda diaria a nivel de artículo, procesará registros de ventas capturados mediante carga de imágenes y empleará un motor analítico que —por medio de un dashboard interactivo y un chatbot conversacional— ofrecerá indicadores clave (KPIs) y recomendaciones operativas. El MVP contempla un módulo de predicción, visualización de métricas, ingestión inteligente de imágenes, alertas de sobreproducción o faltantes y comparaciones semana a semana.

Además de sus aportes técnicos, la propuesta busca contribuir al \emph{Objetivo de Desarrollo Sostenible 12} (Producción y Consumo Responsables) mediante la minimización de desperdicios y el uso eficiente de recursos. Al acercar tecnologías de IA a actores tradicionalmente rezagados en digitalización, se espera profesionalizar la gestión operativa, optimizar la compra de insumos y fortalecer la competitividad de cientos de unidades comerciales que conforman el tejido económico local.

\section{Objetivo General}

Desarrollar una plataforma web de predicción con inteligencia artificial para optimizar la producción diaria y reducir el desperdicio de productos perecederos en comercios, ajustando la oferta a la demanda real mediante datos históricos en el contexto de Buenos Aires 2025.\\

\noindent\textbf{Objetivos Específicos:}

\begin{itemize}
    \item Desarrollar un sistema de predicción de demanda diaria por producto utilizando inteligencia artificial, considerando variables como historial de ventas, día de la semana, clima y feriados.
    
    \item Implementar un módulo de carga y procesamiento automatizado de registros de venta, a partir de imágenes, mediante tecnologías de visión artificial y modelos de lenguaje.
    
    \item Diseñar una interfaz de backoffice con visualización de indicadores clave (KPIs), como márgenes, costos, productos más vendidos y ventas por franja horaria.
    
    \item Incorporar un asistente conversacional basado en lenguaje natural, que facilite la interacción del usuario con el sistema, permitiendo consultas operativas y recomendaciones automatizadas.
    
    \item Integrar un sistema de alertas y recomendaciones accionables, que anticipe sobreproducción o faltantes de stock y sugiera ajustes de planificación.
    
\end{itemize}

\section{Alcance}

El desarrollo de este producto de software será una aplicación web que incluirá las siguientes funcionalidades:

\begin{itemize}
    \item Se desarrollará un módulo de predicción de demanda diaria por producto, utilizando como variables de entrada el historial de ventas, día de la semana, clima y feriados, obteniendo como salida una predicción sobre la producción del producto. 
    \item Se incluirá un dashboard que permita visualizar los KPIs más relevantes para la organización, tales como márgenes de ganancia, costos de insumos, productos más vendidos y visualización de ventas por franja horaria. 
    \item Se implementará un chatbot que funcionará como interfaz principal de interacción con el sistema en el cual el usuario podrá realizar consultas.
    \item Se implementará un sistema de carga de imágenes por parte del usuario, permitiendo subir registros de ventas directamente desde la interfaz web, extrayendo y almacenando automáticamente los datos desde las imágenes mediante LLMs.
    \item Se implementará un conjunto de alertas básicas automatizadas, que advertirán al usuario en casos de potencial sobreproducción o faltantes de stock antelación, en base a la comparación entre los valores proyectados y los datos de producción ingresados. 
    \item Se incorporará una herramienta de comparación de ventas semana a semana, que facilitará la identificación de patrones de comportamiento del consumo y permitirá evaluar la evolución del desempeño comercial en el tiempo.
\end{itemize}

\chapter{Antecedentes}\label{chapter02}

La investigación sobre predicción de demanda para productos perecederos ha avanzado con rapidez en los últimos años gracias a la IA. En el ámbito del comercio de alimentos frescos en línea, se ha demostrado que la incorporación de variables como el clima y el calendario mejora significativamente la precisión de los modelos predictivos (Ni et al., 2022). Posteriormente, investigaciones más recientes en el sector minorista alimentario evidencian que modelos basados en redes neuronales recurrentes, como LSTM, superan sistemáticamente a los enfoques estadísticos tradicionales al minimizar tanto los faltantes como los excesos de inventario (Nassibi et al., 2023). Estos avances reflejan el papel creciente de la inteligencia artificial en la optimización de la gestión de productos perecederos. Aun así, en Argentina se sigue desaprovechando el 12,5\% de la producción agroalimentaria, de las que solo el 1,2\% se pierden en la etapa comercial. El impacto se vuelve crítico en rubros como panaderías y pastelerías, donde en 2024 cerraron más de 400 locales y las ventas de pan cayeron un 53\%.


\section{Marco teórico}

En esta sección se describen los fundamentos conceptuales y técnicos necesarios para comprender el enfoque del sistema propuesto, incluyendo tecnologías de predicción, Inteligencia artificial, aprendizaje automático y su aplicación en la predicción de demanda, y problemáticas sociales vinculadas al desperdicio de alimentos, que son relevantes para el sistema a desarrollar.


\subsection{Inteligencia artificial}

La inteligencia artificial (en adelante IA) es un campo de estudio dentro de la informática que busca desarrollar sistemas capaces de realizar tareas que normalmente requieren inteligencia humana, tales como el razonamiento, la percepción, la toma de decisiones o el aprendizaje. El término fue acuñado formalmente en 1956 durante la conferencia de Dartmouth, considerada el punto de partida de la disciplina \parencite{mccarthy1955}.\\

En términos generales, la IA puede dividirse en dos grandes enfoques:

\begin{itemize}
    \item \textbf{IA débil} (\textit{narrow AI}): especializada en tareas concretas como traducción automática, recomendación de productos o reconocimiento de imágenes \parencite{russell2021}.
    
    \item \textbf{IA fuerte} (\textit{general AI}): de carácter hipotético, orientada a replicar el razonamiento humano en su totalidad \parencite{russell2021}.
\end{itemize}

Gracias al aumento en la capacidad de cómputo y al acceso a grandes volúmenes de datos, el desarrollo reciente de IA basada en aprendizaje automático (\textit{machine learning}) ha permitido avances significativos en múltiples industrias, incluyendo salud, finanzas, logística y comercio minorista \parencite{jordan2015}.

Estos avances han dado lugar al uso extendido de IA en tareas de predicción de demanda, optimización de recursos y automatización de procesos, tal como se propone en el presente proyecto.

La IA también se ha convertido en una herramienta clave para abordar problemáticas socioambientales, como el desperdicio de alimentos, al permitir el análisis en tiempo real de patrones de consumo y la generación de alertas y recomendaciones \parencite{rolnick2019}.

\subsection{Aprendizaje automático}

El aprendizaje automático (\textit{machine learning}, en adelante ML) es una subárea de la inteligencia artificial que se enfoca en desarrollar algoritmos capaces de aprender automáticamente a partir de datos, identificar patrones y realizar predicciones o tomar decisiones sin ser programados explícitamente para cada tarea específica \parencite{mitchell1997}.

En contraste con la programación tradicional —donde se define explícitamente cada regla—, en ML los sistemas ajustan su comportamiento a partir de ejemplos y experiencia previa, lo que permite adaptarse a entornos dinámicos o impredecibles. Existen tres grandes categorías de aprendizaje automático:

\begin{itemize}
    \item \textbf{Supervisado}: el modelo se entrena con datos etiquetados, es decir, conjuntos de datos en los que cada entrada está asociada a una salida conocida. (por ejemplo, ventas históricas con cantidades reales).
    
    \item \textbf{No supervisado}: busca patrones o agrupamientos en datos no etiquetados.
    
    \item \textbf{Por refuerzo}:  el agente aprende a tomar decisiones mediante un sistema de recompensas y penalizaciones asociadas a sus acciones dentro de un entorno. \parencite{sutton2018}.
\end{itemize}

En el ámbito de la predicción de demanda y la gestión comercial, el ML ha demostrado ser altamente efectivo. Algoritmos como regresiones, árboles de decisión, \textit{random forest}, redes neuronales o XGBoost permiten prever comportamientos complejos en entornos de alta variabilidad, como el retail alimentario o los productos perecederos \parencite{carbonneau2008}.

Gracias a su capacidad para adaptarse a datos históricos y variables externas —como el clima, promociones o eventos locales—, el ML se ha convertido en una herramienta estratégica para reducir mermas, optimizar la producción y tomar decisiones basadas en evidencia, alineándose con los objetivos del presente proyecto.

\subsection{Series temporales}

Una serie temporal es una secuencia de observaciones recolectadas en intervalos regulares a lo largo del tiempo. Este tipo de datos permite analizar la evolución de un fenómeno y modelar su comportamiento futuro mediante técnicas estadísticas o de aprendizaje automático \parencite{chatfield2004}. En el caso del comercio minorista, las ventas históricas diarias o semanales de un producto conforman una serie temporal típica.\\

Las series temporales suelen contener tres componentes principales:

\begin{itemize}
    \item \textbf{Tendencia (\textit{trend})}: dirección general del movimiento a largo plazo.
    
    \item \textbf{Estacionalidad (\textit{seasonality})}: patrones repetitivos dentro de un periodo (como días de la semana o estaciones).
    
    \item \textbf{Ruido (\textit{noise})}: variaciones aleatorias e impredecibles.
\end{itemize}

La predicción de series temporales es crucial para aplicaciones como la planificación de la producción, gestión de inventarios y optimización de recursos, ya que permite anticiparse a picos o caídas en la demanda \parencite{hyndman2018}.

Tradicionalmente, se utilizaron modelos como ARIMA, Holt-Winters y modelos exponenciales suavizados, aunque en la actualidad han ganado terreno los modelos basados en aprendizaje automático y redes neuronales, especialmente cuando se incorporan variables exógenas como el clima o eventos especiales \parencite{bandara2020}.

En el presente proyecto, las series temporales constituyen la estructura central sobre la cual se apoyará la generación de predicciones de demanda, mediante el uso de modelos ya existentes y herramientas que permiten adaptarse dinámicamente a los patrones observados en los datos históricos y en el contexto operativo del comercio.

\subsection{Modelos de predicción}

Los modelos de predicción son herramientas matemáticas o computacionales que permiten estimar el valor futuro de una variable en función de sus observaciones pasadas y/o de otras variables relacionadas. En el contexto de series temporales, estos modelos se utilizan para anticipar comportamientos futuros de fenómenos que evolucionan en el tiempo, como la demanda de productos perecederos en comercios minoristas \parencite{hyndman2018}.

Entre los enfoques tradicionales más utilizados se encuentran:

\begin{itemize}

    \item \textbf{Regresión}: es una técnica utilizada en estadística y aprendizaje automático que permite modelar y predecir el valor de una variable dependiente en función de una o más variables independientes. Su objetivo principal es identificar y cuantificar la relación existente entre dichas variables.

    \item \textbf{Regresión lineal}: es la forma más básica de regresión, que asume una relación lineal entre las variables. Modela esta relación mediante una recta que minimiza la diferencia entre los valores observados y los valores predichos. Es especialmente útil en contextos donde los datos presentan una estructura simple y relaciones directas. Sin embargo, su capacidad para capturar patrones complejos o no lineales es limitada, lo que restringe su aplicación en problemas con dinámicas más sofisticadas.
    
    \item \textbf{ARIMA (AutoRegressive Integrated Moving Average)}: combina componentes autorregresivos, de promediado móvil y diferenciación para tratar series no estacionarias \parencite{box2015}.
    
    \item \textbf{Holt-Winters}: extensión del suavizado exponencial que incorpora estacionalidad y tendencia, útil para pronósticos a corto plazo con ciclos estables.
\end{itemize}

Estos modelos son apreciados por su simplicidad y bajo costo computacional, aunque presentan limitaciones cuando hay muchas variables externas o relaciones no lineales.

Con el auge del aprendizaje automático, se han incorporado técnicas más complejas que permiten capturar relaciones no lineales y variables exógenas:

\begin{itemize}
    \item \textbf{Prophet}: modelo desarrollado por Facebook que permite ajustar estacionalidades múltiples y eventos especiales de forma flexible \parencite{taylor2018}.
    
    \item \textbf{LSTM (Long Short-Term Memory)}: tipo de red neuronal recurrente capaz de aprender dependencias de largo plazo, muy utilizada en predicción de series temporales con datos ruidosos o variables múltiples (Hewamalage, 2021).
\end{itemize}

Como se observa en la Figura~\ref{fig:lstm}, esta arquitectura permite mantener y transmitir información a lo largo del tiempo, lo cual la hace especialmente útil para modelar secuencias con relaciones temporales complejas.

\begin{figure}[t]
    \centering
    \includegraphics[width=0.7\textwidth]{images/lstm.png}
    \caption{LSTM (Long Short-Term Memory) básico.}
    \label{fig:lstm}
\end{figure}

\begin{itemize}
    \item \textbf{Transformers}: arquitectura inicialmente desarrollada para procesamiento de lenguaje natural, que ha comenzado a aplicarse con éxito en predicción multivariada de series temporales \parencite{li2019}.
\end{itemize}

La elección del modelo dependerá del tipo de datos disponibles, la granularidad deseada y los requisitos de interpretabilidad y precisión. En este proyecto, estos modelos permitirán estimar con mayor exactitud la demanda diaria de productos, reduciendo mermas y ajustando la producción a las condiciones reales del entorno.

\subsection{Evaluación de modelos predictivos}

La evaluación de modelos predictivos es un paso fundamental en el desarrollo de soluciones basadas en inteligencia artificial o series temporales, ya que permite medir cuán precisas y útiles son las predicciones generadas. Para problemas de regresión —como la predicción de demanda— se utilizan principalmente métricas de error que comparan los valores predichos con los observados.

Entre las más utilizadas se encuentran:

\begin{itemize}
    \item \textbf{MAE (Mean Absolute Error)}: mide el promedio de los errores absolutos entre las predicciones $\hat{y}_t$ y los valores reales $y_t$ \parencite{willmott2005}.
    
    \begin{align}
        \text{MAE} = \frac{1}{n} \sum_{t=1}^{n} \left| y_t - \hat{y}_t \right|
    \end{align}

    \item \textbf{RMSE (Root Mean Squared Error)}: calcula la raíz cuadrada del promedio de los errores al cuadrado, penalizando más fuertemente los errores grandes \parencite{chai2014}.
    
    \begin{align}
        \text{RMSE} = \sqrt{ \frac{1}{n} \sum_{t=1}^{n} \left( y_t - \hat{y}_t \right)^2 }
    \end{align}

    \item \textbf{MAPE (Mean Absolute Percentage Error)}: expresa el error como un porcentaje del valor real. Es útil para comparar modelos en distintas escalas, aunque puede ser inestable cuando $y_t \approx 0$ \parencite{myttenaere2016}.
    
    \begin{align}
        \text{MAPE} = \frac{100}{n} \sum_{t=1}^{n} \left| \frac{y_t - \hat{y}_t}{y_t} \right|
    \end{align}
\end{itemize}

La elección de la métrica depende del contexto. El MAE es más interpretable para usuarios no técnicos, mientras que el RMSE resulta más sensible a errores significativos. 

Estas herramientas también permiten monitorear el desempeño del modelo en producción, facilitando el mantenimiento de su precisión a lo largo del tiempo.

\subsection{Reconocimiento óptico de caracteres}

\indent El Reconocimiento Óptico de Caracteres (OCR) es una tecnología que permite convertir texto impreso o manuscrito en imágenes digitales en texto editable y procesable por máquinas. Es ampliamente utilizada en tareas como la digitalización de documentos, la lectura automatizada de facturas, tickets, formularios y libros (Smith, 2007).

\indent Inicialmente, los sistemas OCR se basaban en técnicas de procesamiento de imágenes y plantillas estáticas, lo que limitaba su capacidad para adaptarse a diferentes formatos, caligrafías o niveles de ruido. Sin embargo, los avances en aprendizaje profundo han revolucionado esta tecnología: los modelos actuales, como CRNN (Convolutional Recurrent Neural Network) y los basados en transformers como TrOCR, ofrecen una mayor precisión y robustez en entornos no estructurados, como fotos tomadas con teléfonos móviles o documentos parcialmente dañados (Baek et al., 2019; Li et al., 2021).

\indent El OCR es especialmente valioso en contextos donde los usuarios no pueden o no desean realizar carga manual de datos, como en pequeños comercios. Al automatizar la lectura de registros de venta o remitos mediante fotos, se reduce la carga operativa y se mejora la calidad del dato ingresado (Khandelwal et al., 2020).

\indent En este proyecto, el OCR cumple un rol clave al permitir a los comerciantes digitalizar información histórica de ventas sin conocimientos técnicos, usando simplemente una fotografía desde la plataforma web, lo cual democratiza el acceso a herramientas de predicción.


\subsection{Modelos de lenguaje extensos}

\indent Los Modelos de Lenguaje Extensos (Large Language Models, LLM) son un tipo avanzado de modelo de IA entrenado con enormes cantidades de texto para comprender, generar y manipular lenguaje natural de forma coherente y contextual. Se basan en arquitecturas como transformers, que han demostrado ser altamente eficaces para tareas complejas de procesamiento del lenguaje natural (Vaswani et al., 2017).

\indent Los LLM aprenden a predecir la siguiente palabra en una secuencia, lo que les permite ejecutar tareas como generación de texto, traducción, resumen automático, respuestas a preguntas y análisis semántico. Modelos como GPT-3, PaLM, BERT o LLaMA han alcanzado niveles de rendimiento cercanos al humano en una variedad de benchmarks lingüísticos (Brown et al., 2020; OpenAI, 2023).

\indent Una de las aplicaciones emergentes de los LLM es su integración con otros flujos de datos no estructurados, como imágenes o documentos escaneados, lo que los convierte en una herramienta poderosa para la automatización de tareas que antes requerían intervención humana. En combinación con OCR y pipelines de extracción, los LLM pueden interpretar textos ambiguos y corregir errores, almacenando la información en forma de tensores (Touvron et al., 2023).

\indent En el contexto de este proyecto, los LLM permiten automatizar la carga de registros de ventas a partir de imágenes, interpretar consultas del usuario en lenguaje natural a través de un chatbot y ofrecer respuestas explicativas, todo sin requerir conocimientos técnicos del usuario final.

\subsection{Chatbots conversacionales}

Los chatbots conversacionales son sistemas informáticos diseñados para interactuar con los usuarios mediante lenguaje natural, ya sea por texto o voz. Su objetivo principal es simular una conversación humana para asistir, informar, resolver consultas o ejecutar acciones específicas dentro de una plataforma digital \parencite{radziwill2017}.

Originalmente basados en reglas fijas y árboles de decisión, los chatbots han evolucionado significativamente gracias a los avances en procesamiento de lenguaje natural (PLN), generación de lenguaje natural (GLN) y, más recientemente, en modelos de lenguaje extensos (or Large Language Models,LLM). Estas tecnologías les permiten comprender intenciones y entidades, responder con mayor coherencia y adaptarse al contexto de las interacciones \parencite{shum2018}.

Existen distintos tipos de chatbots, entre ellos:

\begin{itemize}
    \item \textbf{Chatbots basados en reglas}: operan sobre flujos de conversación predefinidos.
    
    \item \textbf{Chatbots con IA}: utilizan algoritmos de aprendizaje automático, PLN y GLN para entender y generar respuestas dinámicas.
    
    \item \textbf{Chatbots híbridos}: combinan ambos enfoques, lo que los hace flexibles y eficientes.
\end{itemize}

En aplicaciones empresariales, los chatbots se utilizan cada vez más para consultas sobre datos, gestión de operaciones y soporte a decisiones, especialmente en entornos con usuarios no técnicos. Estudios recientes demuestran que la incorporación de asistentes conversacionales mejora la adopción de sistemas analíticos complejos, reduce barreras de entrada y mejora la experiencia del usuario \parencite{knote2021}.

En este proyecto, el chatbot actuará como interfaz principal del sistema, permitiendo al comerciante consultar predicciones de demanda, indicadores clave (KPIs), alertas de sobreproducción y sugerencias de acción, sin necesidad de navegar por menús complejos ni interpretar gráficos técnicos.

\subsection{Bases de datos relacionales y vectoriales}

Las bases de datos relacionales (RDB, por sus siglas en inglés) son estructuras de almacenamiento que organizan la información en tablas con filas y columnas, siguiendo un modelo basado en el álgebra relacional, propuesta por Edgar F. Codd en 1970. Este modelo formaliza las operaciones sobre conjuntos de datos utilizando el concepto de relación, y se convirtió en el estándar dominante para almacenar información estructurada en sistemas informáticos \parencite{codd1970}. 

Las bases relacionales permiten realizar consultas complejas mediante lenguajes como SQL y son ampliamente utilizadas en aplicaciones empresariales debido a su robustez, integridad referencial y facilidad de acceso \parencite{coronel2020}.

En contraposición, las bases de datos vectoriales son un tipo más reciente de almacenamiento orientado a datos no estructurados o semiestructurados representados como tensores. Su principal aplicación está en sistemas que utilizan inteligencia artificial, especialmente modelos de lenguaje, visión por computadora y recuperación semántica, donde se requiere comparar elementos no por igualdad exacta, sino por similitud de contexto o significado \parencite{johnson2019}.

Estos vectores se obtienen comúnmente a través de \textit{embeddings} generados por modelos como Word2Vec, BERT o CLIP, y se almacenan en motores especializados como FAISS, Milvus o Pinecone, optimizados para búsquedas por proximidad (\textit{nearest neighbor search}).

En el contexto de este proyecto, las bases relacionales serán utilizadas para almacenar información estructurada como productos, ventas, fechas y predicciones, mientras que las bases vectoriales podrían emplearse en versiones futuras para mejorar el rendimiento del chatbot, permitiéndole recuperar respuestas basadas en similitud semántica entre preguntas y registros históricos o documentación del sistema.

\subsection{Desperdicio de alimentos}

El desperdicio de alimentos hace referencia a la pérdida de productos aptos para el consumo humano que son descartados, deteriorados o no utilizados en etapas finales de la cadena alimentaria, como la comercialización, el almacenamiento o el consumo doméstico. Se diferencia de la pérdida de alimentos, que ocurre en fases anteriores como la producción, poscosecha o procesamiento (FAO, 2019).

A nivel mundial, se estima que alrededor del 17\% de los alimentos disponibles para los consumidores se desperdicia, lo que representa no solo un problema ético y de seguridad alimentaria, sino también una amenaza ambiental por el uso innecesario de recursos como agua, tierra y energía, y la generación de gases de efecto invernadero \parencite{unep2021}.

En Argentina, se pierde o desperdicia el 12{,}5\% de la producción agroalimentaria, lo que equivale a aproximadamente 16 millones de toneladas por año, con consecuencias económicas directas para todos los actores de la cadena de valor \parencite{tiscornia2022}. Particularmente en supermercados, autoservicios y comercios de proximidad, estudios recientes estiman que las pérdidas en categorías frescas superan el 3\% de la facturación, debido a sobreproducción, quiebres de stock, manipulación inadecuada o falta de planificación \parencite{weteam2021}.

Estas cifras revelan la importancia de implementar soluciones basadas en inteligencia artificial y analítica de datos para alinear la producción con la demanda real, evitar el sobrante no comercializable y aumentar la eficiencia operativa. Además, el combate al desperdicio de alimentos contribuye directamente al cumplimiento del Objetivo de Desarrollo Sostenible (ODS) 12: Producción y Consumo Responsables, propuesto por la ONU para 2030.

\subsection{Impacto económico}

El desperdicio de alimentos no solo representa una pérdida de recursos naturales y energéticos, sino también un impacto económico significativo para productores, distribuidores, comercios y consumidores. En cada etapa de la cadena alimentaria, los alimentos descartados implican costos hundidos en insumos, mano de obra, energía, logística y espacio, sin retorno económico alguno \parencite{gustavsson2011}.

A nivel global, se estima que el valor económico del desperdicio asciende a 1 billón de dólares anuales, afectando tanto a países desarrollados como en desarrollo (FAO, 2013). En entornos urbanos y comerciales, las pérdidas se concentran especialmente en productos frescos y perecederos —como panificados, frutas, carnes y lácteos— debido a errores de planificación, sobreproducción, fluctuaciones de demanda o una gestión ineficiente de inventarios (FAO, 2019).

En Argentina, estudios sectoriales revelan que las pérdidas en supermercados y autoservicios superan el 3\% de la facturación en categorías de frescos y almacén, representando un margen crítico para negocios con alta rotación y baja rentabilidad unitaria \parencite{weteam2021}. En comercios pequeños, como panaderías y pastelerías, esta ineficiencia se ve agravada por la falta de herramientas analíticas y predictivas que permitan anticipar la demanda real, provocando mermas frecuentes y subutilización de insumos.

El impacto económico también afecta a escala sistémica: cada tonelada de alimento desperdiciado implica no solo una pérdida directa, sino también costos ocultos en la cadena logística, residuos, tratamiento y emisiones \parencite{refed2016}. Desde esta perspectiva, el desarrollo de soluciones tecnológicas que minimicen el desperdicio contribuye no solo a mejorar la rentabilidad del negocio, sino también a reducir costos operativos y ambientales a largo plazo.

\subsection{Impacto ambiental}

El desperdicio de alimentos tiene consecuencias ambientales severas, ya que cada unidad de alimento producida pero no consumida implica un uso innecesario de recursos naturales, como agua, suelo y energía, además de generar residuos y emisiones contaminantes. La huella ambiental del desperdicio incluye tres grandes dimensiones: la huella hídrica, la huella de carbono y el uso del suelo \parencite{fao2013}.

A nivel mundial, se calcula que el desperdicio de alimentos genera entre el 8\% y el 10\% de las emisiones globales de gases de efecto invernadero (GEI), lo que equivale a más de 3.300 millones de toneladas de CO\textsubscript{2} emitidas anualmente \parencite{unep2021}. 

En cuanto a recursos hídricos, se estima que alrededor de 250~km\textsuperscript{3} de agua se utilizan cada año para producir alimentos que nunca serán consumidos, lo que representa una presión significativa sobre acuíferos y sistemas hídricos vulnerables. Además, el uso ineficiente del suelo para cultivos que luego se pierden contribuye a la deforestación, pérdida de biodiversidad y degradación de ecosistemas \parencite{kummu2012}.

En el caso de Argentina, donde se desperdician más de 16 millones de toneladas de alimentos por año \parencite{tiscornia2022}, el impacto ambiental se agrava por la dependencia del sector agroindustrial como motor económico, lo que exige soluciones que equilibren productividad y sostenibilidad. Reducir el desperdicio, por tanto, no solo mejora la eficiencia de los sistemas alimentarios, sino que también representa una estrategia concreta de mitigación del cambio climático \parencite{unep2021}.

Desde esta perspectiva, el proyecto presentado cobra relevancia al buscar reducir la sobreproducción mediante herramientas de predicción basadas en inteligencia artificial, contribuyendo así a una gestión más sostenible de los recursos.

\subsection{ODS 12}

El Objetivo de Desarrollo Sostenible 12 (ODS 12), propuesto por las Naciones Unidas en la Agenda 2030, tiene como meta garantizar modalidades de consumo y producción sostenibles, promoviendo un uso eficiente de los recursos, la energía y los sistemas productivos, sin comprometer las necesidades de las generaciones futuras \parencite{onu2015}.

Uno de los ejes centrales del ODS 12 es la reducción sustancial del desperdicio de alimentos, tanto en la etapa de producción como en los sectores minorista y de consumo. La meta 12.3, en particular, establece como compromiso internacional reducir a la mitad el desperdicio de alimentos per cápita mundial en el comercio minorista y el consumidor para el año 2030, y disminuir las pérdidas a lo largo de las cadenas de producción y suministro \parencite{unep2021}.

Este objetivo reconoce que el desperdicio alimentario es un problema transversal: tiene impactos económicos, sociales (inseguridad alimentaria) y ambientales (emisiones de gases de efecto invernadero, uso de agua y suelo). Por eso, la tecnología juega un rol clave como habilitadora de soluciones innovadoras y escalables, particularmente en el sector privado.

En este contexto, el presente proyecto de ingeniería se alinea directamente con el ODS 12, al proponer una plataforma que utiliza inteligencia artificial para predecir la demanda real en comercios de productos perecederos, evitando sobreproducción y mermas innecesarias. Así, no solo mejora la rentabilidad de los negocios, sino que contribuye a un modelo de producción más eficiente, consciente y responsable con el entorno.

\subsection{Digitalización y brecha tecnológica en PYMEs}

La digitalización implica la incorporación de tecnologías digitales en los procesos productivos, administrativos y comerciales de las organizaciones, con el objetivo de mejorar su eficiencia, competitividad y capacidad de adaptación. Sin embargo, las pequeñas y medianas empresas (PYMEs) enfrentan múltiples barreras estructurales que dificultan su transformación digital, generando una brecha tecnológica creciente respecto de grandes empresas o corporaciones \parencite{oecd2021}.

Entre las principales limitaciones que afectan a las PYMEs se encuentran:

\begin{itemize}
    \item Falta de acceso a infraestructura tecnológica adecuada (hardware, software, conectividad).
    \item Baja disponibilidad de talento digital o personal capacitado.
    \item Costos percibidos como elevados, especialmente en soluciones basadas en IA o big data.
    \item Desconocimiento de herramientas existentes y sus beneficios.
    \item Miedo al cambio o resistencia organizacional \parencite{jordao2022}.
\end{itemize}

En América Latina y particularmente en Argentina, estas brechas son aún más pronunciadas. Estudios del BID señalan que solo 1 de cada 4 PYMEs en la región adopta tecnologías digitales avanzadas, y que más del 60~\% se encuentra en niveles básicos de digitalización \parencite{bid2020}. Complementariamente, un estudio reciente sobre digitalización contable en PYMEs de Ecuador identificó que los principales obstáculos son la falta de capacitación, los altos costos iniciales y la resistencia al cambio, y destacó que el acceso a tecnología, formación profesional y apoyo institucional son determinantes para una adopción exitosa \parencite{vasconez2025}.

Entre sus hallazgos, se estima que la digitalización puede aumentar los ingresos en un 30~\% y reducir los costos operativos en un 20~\%, aunque aún persisten desafíos relacionados con la visión estratégica y el capital disponible.

En el sector alimentario, la falta de herramientas tecnológicas específicas impide a muchos comercios ajustar su producción a la demanda real, generando ineficiencias como sobreproducción, desperdicio o rotura de stock. Por eso, el desarrollo de plataformas accesibles, intuitivas y enfocadas en resolver problemas concretos de gestión —como la que se propone en este proyecto— resulta clave para acortar la brecha digital, profesionalizar la toma de decisiones y mejorar la sostenibilidad operativa de los pequeños comercios.

\newpage % <-- salto de página

\section{Estado del arte}

La predicción de demanda en comercios que operan con productos perecederos representa un desafío significativo debido a la naturaleza volátil y sensible al tiempo de estos bienes. En las últimas décadas, el avance de las tecnologías de inteligencia artificial y aprendizaje automático ha permitido desarrollar soluciones cada vez más sofisticadas para anticipar patrones de consumo y optimizar la producción, con el objetivo de minimizar pérdidas y mejorar la eficiencia operativa.

Este estado del arte presenta una revisión crítica de las soluciones tecnológicas existentes y los enfoques académicos más relevantes aplicados a la predicción de demanda en el sector alimentario, con especial énfasis en productos frescos y perecederos. Se analizan modelos, métricas y tecnologías empleadas, así como las limitaciones que enfrentan estas soluciones, especialmente en contextos de pequeñas y medianas empresas con restricciones operativas y tecnológicas.

A partir de este análisis, se evidencian las oportunidades y necesidades no cubiertas que motivan el desarrollo de la propuesta de este proyecto, orientada a un contexto local con condiciones económicas y sociales específicas, buscando ofrecer una herramienta accesible, flexible y efectiva para comercios minoristas.


\subsection{Soluciones tecnológicas existentes}

En el mercado global existen diversas soluciones tecnológicas orientadas a la predicción de demanda, particularmente en retail alimentario:

\begin{itemize}
    \item \textbf{SAP Forecasting and Replenishment (F\&R)}: Es una herramienta empresarial enfocada en la planificación de inventarios y la reposición automática. Utiliza modelos estadísticos y reglas de negocio, pero no incluye inteligencia artificial. Para eso, SAP ofrece otra solución aparte llamada, \textit{SAP Predictive Replenishment}, orientada a clientes con mayores requerimientos tecnológicos.
    
    \item \textbf{Oracle Retail Demand Forecasting}: Permite hacer pronósticos utilizando algoritmos de aprendizaje automático y tiene en cuenta factores como estacionalidad, promociones o eventos. Es una solución muy completa, pero está pensada para grandes empresas y requiere una inversión importante para su implementación, lo que la hace poco viable para PYMEs.
    
    \item \textbf{BakePlan}: Es una herramienta pensada específicamente para panaderías, especialmente en Europa. Estima la cantidad ideal de productos horneados por día basándose en ventas pasadas y condiciones climáticas. Si bien logra buenos resultados en reducción de merma, su aplicación está limitada a contextos muy estructurados, y no se adapta fácilmente a entornos informales o menos digitalizados.
    
    \item \textbf{Blue Yonder Luminate}: es una plataforma en la nube que utiliza inteligencia artificial para realizar predicciones de demanda. Está dirigida principalmente a grandes cadenas y supermercados, pero no es una opción adecuada para pequeños comercios, ya que requiere cierto grado de digitalización previa y personal técnico capacitado para su uso.
\end{itemize}

Estos estudios confirman que los modelos neuronales como LSTM y Bi-LSTM superan en precisión a los enfoques clásicos, especialmente en entornos con múltiples variables externas (como el clima), lo que refuerza su aplicabilidad en el presente proyecto.

\subsection{Estudios y enfoques académicos}

Investigaciones recientes han abordado la predicción de demanda para alimentos perecederos con distintos enfoques:

\begin{itemize}
    \item Se han comparado distintos modelos de \textit{machine learning} para la predicción de demanda en la industria alimentaria, como Random Forest, SVM y LSTM. En estas pruebas, el modelo LSTM fue el que logró el menor error promedio (RMSE), al demostrar una mejor capacidad para capturar patrones temporales no lineales \parencite{nassibi2023}.

    \item En el ámbito del \textit{e-commerce} de alimentos frescos, se ha utilizado un modelo Bi-LSTM para predecir la demanda logística. Al incorporar variables climáticas y festivos, se logró reducir el MAE en un 12{,}6\,\% respecto de los modelos base \parencite{ni2022}.

    \item También se ha propuesto el modelo LSTM-MSNet para predecir múltiples series temporales con patrones estacionales. Este enfoque superó a modelos tradicionales como Prophet y Holt-Winters, especialmente en escenarios con datos ruidosos o incompletos \parencite{bandara2020}.
\end{itemize}

Estos estudios confirman que los modelos neuronales como LSTM y Bi-LSTM superan en precisión a los enfoques clásicos, especialmente en entornos con múltiples variables externas (como el clima), lo que refuerza su aplicabilidad en el presente proyecto.



\begin{table}[t]
    \centering
    \renewcommand{\arraystretch}{1.3}
    \caption{Comparación de enfoques}
    \label{tab:comparacion}
    \begin{tabular}{|p{2.9cm}|p{2cm}|p{3cm}|p{5cm}|}
        \hline
        \textbf{Solución} & \textbf{Tipo} & \textbf{Tecnologías principales} & \textbf{Limitaciones destacadas} \\
        \hline
        SAP F\&R & Comercial & Suite de reposición SAP; forecasting integrado; integración ERP & Alto costo de licencias y consultoría; requiere datos históricos limpios y equipo IT; tiempos de implantación largos \\
        \hline
        Oracle RDF & Comercial  & Oracle Retail; forecast multinivel; planogramas & Complejidad técnica; vendor lock-in; demanda gobierno de datos sólido; implantación 4–9 meses \\
        \hline
        BakePlan & Comercial  & Planificación de producción/pedidos; reglas predefinidas & Personalización limitada; sin señales externas (clima/eventos); escalabilidad acotada \\
        \hline
        POS + add-ons (locales) & Comercial  & Reportes de ventas/stock; integraciones básicas & Sin predicción; sin recomendaciones operativas; no considera clima/eventos; foco administrativo \\
        \hline
        Excel / BI genérico & Herramienta & Dashboards y/o manuales & Dependencia de trabajo manual; errores humanos; sin modelos predictivos ni automatización operativa \\
        \hline
        \textbf{AIVA (PFI)} & \textbf{Comercial} & \textbf{Predicción con clima y eventos; sugerencias de producción; carga asistida OCR/LLM} & Requiere calibración inicial y adopción; datos iniciales ruidosos \\
        \hline
    \end{tabular}
\end{table}



\subsection{Aporte diferencial del proyecto}

El sistema propuesto en este proyecto se distingue de las soluciones analizadas por las siguientes características:

\begin{itemize}
    \item \textbf{Accesibilidad}: pensado específicamente para pequeños comercios argentinos sin sistemas previos de gestión ni personal técnico.

    \item \textbf{Multimodalidad}: permite la carga de registros mediante imágenes, utilizando modelos OCR y LLMs, lo cual no está presente en las herramientas comerciales revisadas.

    \item \textbf{Interfaz natural}: reemplaza los paneles tradicionales por un chatbot conversacional en lenguaje natural, facilitando el uso incluso por usuarios no técnicos.

    \item \textbf{Orientación local}: los modelos se entrenan considerando variables específicas como clima y feriados de Buenos Aires en el año 2025.
\end{itemize}


\subsection{Conclusión del estado del arte}
El análisis realizado permite constatar que, si bien existen múltiples soluciones comerciales y académicas orientadas a la predicción de demanda en el sector alimentario, la mayoría presenta limitaciones significativas para su adopción en pequeños comercios. Las herramientas comerciales suelen implicar altos costos, requerimientos técnicos avanzados o infraestructuras previas de gestión, lo que las hace poco viables en entornos de baja digitalización. Por otro lado, los enfoques académicos demuestran un alto potencial predictivo, en particular mediante modelos neuronales como LSTM y Bi-LSTM, pero carecen de implementaciones accesibles y adaptadas al mercado local. En este contexto, AIVA se posiciona como una propuesta diferencial al integrar accesibilidad, multimodalidad y orientación a la realidad de las PYMEs argentinas, aportando una solución práctica y escalable que combina avances tecnológicos recientes con un enfoque inclusivo y sostenible.

\newpage % <-- salto de página

\section{User research}

Con el objetivo de validar la propuesta y diseñar una herramienta que responda a problemáticas reales del entorno operativo, se llevó a cabo una instancia de investigación cualitativa y cuantitativa. La metodología incluyó una encuesta dirigida a usuarios representativos del mercado objetivo y una entrevista en profundidad a una comerciante del rubro de productos perecederos.  

Ambas instancias permitieron identificar necesidades concretas, limitaciones tecnológicas actuales y oportunidades de mejora en los procesos de planificación y producción diaria. A continuación, se presenta un resumen de los principales hallazgos obtenidos a partir de la encuesta y la entrevista realizada.


\subsection{Encuesta}

Con el objetivo de validar y parametrizar AIVA, se llevó a cabo una encuesta anónima a consumidores de panificados. La encuesta releva hábitos de compra y percepciones sobre disponibilidad de productos, así como la influencia del clima, los momentos y días de mayor compra y el tamaño típico de compra. Estos datos constituyen la base empírica para justificar y calibrar las funciones de pronóstico y alertas del sistema. Se obtuvieron 134 respuestas válidas mediante un cuestionario auto-administrado en línea, en el cual se observa una mayor participación de jóvenes (16–24) y de personas de 45–54 años.

Del total de encuestados, alrededor del 69\% compra panificados más de una vez al mes y alrededor del 42\% compra al menos una vez por semana. Esto indican la necesidad de pronósticos con horizonte semanal e intradía y de ajustes de lotes orientados a picos específicos, evitando sobreproducción y faltantes.

\begin{figure}[t]
    \centering
    \includegraphics[width=0.78\textwidth]{images/FrecuenciaCompraPanificados.png}
    \caption{Frecuencia de compra de panificados.}
    \label{fig:frecuencia-compra-panificados}
\end{figure}

La distribución horaria muestra una fuerte concentración durante la tarde. El 53,6\% de las compras se realizan entre 15–18 h, muy por encima de los demás horarios. Este patrón confirma un pico intradiario pronunciado que justifica que AIVA trate la hora como variable de alto peso en el pronóstico e integre interacciones con día de semana y clima; operativamente, habilita recomendaciones de reposición con la suficiente antelación antes del pico, alertas de quiebre y sugerencias de sustitución cuando el stock proyectado sea insuficiente, acciones que atacan el momento de mayor riesgo de ventas perdidas.

\begin{figure}[t]
    \centering
    \includegraphics[width=0.78\textwidth]{images/MomentoDeCompra.png}
    \caption{Momento de compra de panificados.}
    \label{fig:momento-compra}
\end{figure}


El 86,3\% de las personas declara comprar principalmente en panaderías/confiterías de barrio, lo que ubica al canal objetivo de AIVA exactamente donde se concentra la demanda. En ese mismo universo, al menos el 74,1\% experimentó algún faltante en el último mes. Esta combinación del canal dominante y alta incidencia de indisponibilidad, confirma la relevancia operativa del problema y respalda el diseño de AIVA: pronósticos por producto y franja horaria para panaderías de barrio, alertas tempranas de quiebre y sugerencias de sustitución para mitigar pérdidas justo donde más se compra.

\begin{figure}[t]
    \centering
    \includegraphics[width=0.78\textwidth]{images/NoHubieraProducto.png}
    \caption{Situación en la que faltó algún producto en la panadería.}
    \label{fig:no-hubiera-producto}
\end{figure}


Ante la última situación de indisponibilidad, la mayoría reorientó la compra dentro del mismo local: 41,7\% eligió otra variedad/sabor y 28,1\% otro producto. En cambio, 25,9\% constituyó venta perdida. El 14,4\% se fue a otro local y 11,5\% no compró. Este patrón cuantifica tanto la oportunidad de retención como el costo del stockout. Para AIVA, estos datos respaldan un motor de sustituciones en tiempo real (sugerencias por afinidad y disponibilidad), alertas de quiebre y priorización de horneadas de los SKU críticos; además, permiten estimar pérdidas y monitorear la tasa de ventas perdidas como indicador operativo clave.


\begin{figure}[t]
    \centering
    \includegraphics[width=0.78\textwidth]{images/AccionPosterior.png}
    \caption{Acción tomada por la persona al no conseguir el producto deseado.}
    \label{fig:accion-posterior}
\end{figure}


El clima aparece como un factor determinante de la demanda: más de la mitad de las personas declara comprar más en días fríos (58\%) y lluviosos (55\%), mientras que en días calurosos predomina la disminución de compra (46\% “compro menos”). Este patrón confirma que las condiciones meteorológicas inciden de forma directa en el volumen a producir y cuándo hacerlo. En consecuencia, AIVA debe ponderar el clima como predictor de alto peso e integrar pronóstico de temperatura y precipitaciones para reforzar horneadas/reposiciones antes de los picos en frío/lluvia, y moderar lotes (con alertas y sustituciones) en calor, minimizando faltantes y sobreproducción.

\begin{figure}[t]
    \centering
    \includegraphics[width=0.78\textwidth]{images/CompraSegunClima.png}
    \caption{Frecuencia de compras según las condiciones climáticas.}
    \label{fig:compras-segun-clima}
\end{figure}


La evidencia recolectada confirma que el problema de disponibilidad en panificados es real y frecuente. Tres de cada cuatro personas experimentaron algún faltante el último mes y, ante esa situación, alrededor de 1 de cada 4 casos termina en venta perdida. Estos resultados muestran que AIVA ataca un problema real, en el canal correcto y con funcionalidades que, según los datos, reducirán faltantes y mermas sin sobreproducir.


\subsection{Entrevista a Milena, encargada de la confitería ``La Catalana''}

Con el propósito de profundizar en el funcionamiento operativo de negocios dedicados a la elaboración y venta de productos perecederos, se realizó una entrevista a Milena González, encargada de la confitería \textit{La Catalana}, un establecimiento familiar con más de quince años de trayectoria en el rubro de la pastelería artesanal. El objetivo principal de la entrevista fue identificar las prácticas actuales de gestión de producción y demanda, así como los desafíos asociados al control de inventario y desperdicio de productos, con el fin de validar la pertinencia de una herramienta predictiva orientada a este tipo de comercios.

La Catalana cuenta con un equipo de 19 empleados y se especializa en la elaboración de una amplia variedad de productos de panadería y pastelería. Entre los más vendidos se destacan las tortas, los sandwichitos de miga y las facturas, mientras que los bombones y algunas masas finas registran una rotación menor. En promedio, los productos tienen una vida útil de entre dos y tres días: el pan se produce diariamente, y las porciones y masas se conservan durante un máximo de tres días, garantizando frescura pero también imponiendo la necesidad de una planificación precisa.

Actualmente, el negocio no utiliza ningún sistema formal de gestión o predicción. La estimación diaria de producción se realiza de forma empírica, basándose en la experiencia y el conocimiento del comportamiento de los clientes. Milena explicó que las decisiones se toman “intuitivamente”, considerando variables generales como el día de la semana o la proximidad de fines de semana, momentos en los cuales la demanda suele incrementarse.

En cuanto al registro histórico de ventas, La Catalana no cuenta con herramientas que permitan analizar los datos de manera sistemática. Esto representa una limitación relevante, ya que la ausencia de información consolidada impide realizar ajustes estratégicos en la producción y dificulta la detección de patrones de consumo. Milena remarcó la importancia de poder estimar con precisión la demanda, dado que los errores de cálculo generan pérdidas económicas al no poder comercializar a tiempo los productos elaborados.

Milena expresó además un claro interés en incorporar tecnologías que ayuden a optimizar la producción y reducir el desperdicio. Si bien actualmente no disponen de ninguna herramienta predictiva, considera que contar con un sistema que permita proyectar la demanda sería de gran ayuda para mejorar la eficiencia del negocio, evitar pérdidas y garantizar la disponibilidad de productos ante picos de demanda. Sin embargo, enfatizó que cualquier solución tecnológica debería ser “fácil e intuitiva de usar”, ya que el equipo no cuenta con una formación técnica avanzada.

En síntesis, la entrevista a Milena pone de manifiesto la necesidad de herramientas accesibles que faciliten la gestión de la producción en panaderías y confiterías tradicionales. Su testimonio refuerza la relevancia de desarrollar soluciones tecnológicas simples, pero efectivas, que permitan transformar la experiencia empírica de los dueños y encargados en decisiones basadas en datos. Esto valida el enfoque del presente proyecto, orientado a ofrecer una solución de predicción de demanda que contribuya a la eficiencia operativa y sostenibilidad económica de este tipo de negocios.


\subsection{Conclusión del user research}

La combinación de los resultados obtenidos a partir de la encuesta a consumidores y la entrevista en profundidad a una comerciante del rubro permitió validar empíricamente la necesidad y el potencial de una herramienta como AIVA.  

\indent Por un lado, la encuesta evidenció un patrón de consumo regular y predecible en el mercado de panificados, con picos de compra concentrados en franjas horarias específicas y una fuerte dependencia de variables externas como el clima y el día de la semana. Asimismo, se constató una alta frecuencia de faltantes: tres de cada cuatro personas experimentaron la ausencia de algún producto en el último mes, y en una de cada cuatro ocasiones esto derivó en una venta perdida. Este hallazgo cuantifica con claridad la magnitud del problema operativo que AIVA busca resolver: la ineficiencia en la planificación y reposición de productos perecederos.  

\indent Por otro lado, la entrevista con Milena González, encargada de la confitería \textit{La Catalana}, permitió comprender en profundidad las limitaciones cotidianas que enfrentan los comercios del sector. La ausencia de sistemas formales de registro y predicción obliga a basar la producción en la intuición y experiencia del personal, lo que incrementa el riesgo de sobreproducción o quiebre de stock. Sin embargo, la disposición expresada por Milena a incorporar soluciones tecnológicas simples y accesibles refuerza la viabilidad de adopción de AIVA dentro de este tipo de entornos.  

\indent En conjunto, ambos instrumentos de investigación —encuesta y entrevista— confirman que el problema de disponibilidad y desperdicio en panaderías y confiterías es real, frecuente y económicamente relevante. Además, validan la hipótesis central del proyecto: existe una oportunidad concreta para aplicar inteligencia artificial en la predicción de demanda de productos perecederos, generando beneficios medibles en eficiencia operativa, reducción de mermas y aumento de ventas.  

\indent Los hallazgos de esta etapa consolidan la base empírica sobre la cual se diseñará e implementará AIVA, asegurando que su arquitectura funcional y sus algoritmos respondan a las condiciones reales del mercado y a las necesidades específicas de los usuarios finales.


\chapter{Descripción}\label{chapter03}

AIVA es un sistema web orientado a pequeñas y medianas empresas de alimentos perecederos, como por ejemplo panaderías y confiterías, que busca reducir mermas y quiebres de stock mediante la planificación de producción y reposición basada en datos. La propuesta integra, en una única plataforma, la captura flexible de ventas (punto de venta, importación de CSV y carga asistida por imágenes), el enriquecimiento contextual (clima, calendario y patrones semanales) y un módulo de predicción de demanda diaria por producto. Sobre esta base, el sistema ofrece visualizaciones ejecutivas (backoffice), un asistente conversacional para consultas operativas y un mecanismo de alertas que anticipa situaciones de sobreproducción o faltantes. 

El núcleo analítico del sistema combina el historial transaccional con variables exógenas para estimar la demanda esperada a corto plazo. Este módulo implementa un enfoque heurístico transparente centrado en tendencia, estacionalidad y condiciones meteorológicas, con un diseño abierto a la incorporación de modelos estadísticos y de aprendizaje automático más avanzados. La arquitectura prioriza componentes de bajo costo y rápida adopción, asegurando trazabilidad de datos, reproducibilidad de resultados y una experiencia de usuario alineada con los flujos cotidianos del negocio.

\vspace{1cm}


\section{Funcionalidades del sistema}

\begin{itemize}
    \item \textbf{Autenticación y perfiles de usuario}
    \begin{itemize}
        \item Registro e inicio de sesión mediante servicio gestionado (Supabase Auth).
        \item Persistencia de sesión y provisión de identidad a la aplicación.
        \item Gestión básica de perfil de usuario.
    \end{itemize}

    \item \textbf{Configuración y localización}
    \begin{itemize}
        \item Inicialización de la ubicación del usuario (por defecto, Ciudad Autónoma de Buenos Aires).
        \item Persistencia de la ubicación para consumo por servicios externos (clima).
    \end{itemize}

    \item \textbf{Gestión de productos y categorías}
    \begin{itemize}
        \item Crear, leer, actualizar y eliminar productos y categorías.
        \item Carga masiva de productos vía CSV con previsualización y validación básica.
        \item Ajustes manuales de stock y visualización de movimientos de stock.
    \end{itemize}

    \item \textbf{Registro y administración de ventas}
    \begin{itemize}
        \item Punto de Venta (POS): Flujo de carrito, totales y confirmación de venta. Alta de venta y detalle de ítems en base de datos. Enriquecimiento contextual de cada transacción para análisis de demanda.
        \item Importación de ventas históricas (CSV): Carga de archivos CSV con validaciones y normalización.
        \item Carga asistida por imágenes (OCR): Interfaz de carga y estados de procesamiento.
    \end{itemize}

    \item \textbf{Datos contextuales de clima}
    \begin{itemize}
        \item Integración con API meteorológica.
        \item Disponibilización de variables climáticas para análisis y predicción.
    \end{itemize}

    \item \textbf{Análisis de demanda}
    \begin{itemize}
        \item Consolidación de datos transaccionales enriquecidos.
        \item Consultas y visualizaciones exploratorias (tendencias por producto/categoría, comparativas temporales).
    \end{itemize}

    \item \textbf{Predicción de demanda}
    \begin{itemize}
        \item Generación de predicciones: Estimación diaria por producto mediante enfoque heurístico (tendencia, estacionalidad, día de semana y clima). Indicadores asociados a cada predicción.
        \item Persistencia y consulta de predicciones: Almacenamiento de predicciones para trazabilidad y auditoría. Gráficos comparativos entre predicción y ventas observadas.
        \item Predicciones resumidas para tableros: Consolidación de resultados para consumo eficiente por el dashboard.
    \end{itemize}

    \item \textbf{Backoffice (dashboard)}
    \begin{itemize}
        \item Indicadores ejecutivos (ventas, productos destacados, clima, predicciones vigentes).
        \item Visualizaciones de gráficos con filtros para seguimiento operativo.
    \end{itemize}

    \item \textbf{Reportes}
    \begin{itemize}
        \item Generación de reportes con agregados por período, producto y categoría.
        \item Gráficos de evolución y comparativas semana a semana.
    \end{itemize}

    \item \textbf{Alertas operativas}
    \begin{itemize}
        \item Listado de alertas de referencia (p. ej., stock bajo, sobreproducción potencial, feriados).
        \item Priorización de casos para acción correctiva.
    \end{itemize}

    \item \textbf{Asistente conversacional}
    \begin{itemize}
        \item Interfaz de chat para consultas frecuentes y navegación guiada.
        \item Respuestas basadas en reglas; integración con modelos de lenguaje planificada.
    \end{itemize}

    \item \textbf{Gestión de planes y suscripciones}
    \begin{itemize}
        \item Plan gratuito por defecto y soporte para planes pagos.
        \item Asociación de usuarios a planes y consulta del estado de suscripción.
    \end{itemize}

    \item \textbf{Perfil y preferencias}
    \begin{itemize}
        \item Edición de datos básicos de perfil.
        \item Configuraciones generales del usuario.
    \end{itemize}

    \item \textbf{Validaciones y manejo de errores}
    \begin{itemize}
        \item Validación de formularios y archivos (CSV).
        \item Mensajería de estados (carga, éxito, error) y prevención de acciones inválidas.
    \end{itemize}

    \item \textbf{Seguridad y segmentación de datos}
    \begin{itemize}
        \item Acceso autenticado a recursos y filtrado de datos por usuario.
        \item Preparado para políticas de control a nivel base de datos (RLS) en despliegues gestionados.
    \end{itemize}
\end{itemize}

\vspace{1cm}
\section{Tecnologías utilizadas}\label{sec:tecnologias}
A continuación se detallan las tecnologías empleadas en AIVA, explicando su rol dentro de la solución, el modo de integración y los criterios que justifican su elección. La selección prioriza un balance entre rapidez de implementación, bajo costo operativo, mantenibilidad y adecuación a los requerimientos del sistema.

\subsection{Next.js (React + TypeScript)}
Next.js es un \textit{framework} sobre React que incorpora enrutamiento, optimizaciones de rendimiento y opciones de renderizado (cliente/servidor). En AIVA se utiliza para:
\begin{itemize}
    \item Implementar la interfaz web con componentes reutilizables y tipado estático (TypeScript), reduciendo defectos en tiempo de compilación.
    \item Organizar el ruteo (App Router) separando vistas operativas (\textit{dashboard}, POS, predicción, reportes) de páginas auxiliares (autenticación, perfil).
    \item Favorecer una arquitectura mantenible (módulos, \textit{hooks} y servicios) con buena escalabilidad del frontend.
\end{itemize}
\noindent\textbf{Motivación.} Ecosistema maduro y documentación extensa, compatible con librerías modernas de UI.

\subsection{Tailwind CSS y \textit{shadcn/ui}}
Tailwind CSS provee utilidades de estilo de bajo nivel; \textit{shadcn/ui} aporta componentes accesibles basados en Tailwind.
\begin{itemize}
    \item Permiten maquetar formularios, tablas, modales y gráficos con consistencia visual y tiempos de desarrollo acotados.
    \item Estandarizan patrones de interacción (validaciones, estados de carga y error) y facilitan la personalización fina del diseño.
\end{itemize}
\noindent\textbf{Motivación.} Coherencia visual y productividad. 

\subsection{Recharts}
Recharts es una biblioteca de gráficos para React.
\begin{itemize}
    \item Se utiliza en el \textit{dashboard} y reportes para visualizar KPIs (ventas por período, productos destacados) y comparativas (predicción vs.\ observado).
    \item Ofrece componentes responsivos y composables, adecuados para vistas con filtros interactivos.
\end{itemize}
\noindent\textbf{Motivación.} Integración directa con React y bajo costo de aprendizaje.

\subsection{Supabase Auth}
Servicio gestionado para registro, inicio de sesión y manejo de sesiones (incluida verificación por correo).
\begin{itemize}
    \item Centraliza la identidad del usuario y emite \textit{tokens} para acceder a recursos protegidos.
    \item Simplifica la implementación de rutas seguras y del \textit{feature gating} por plan/suscripción.
\end{itemize}
\noindent\textbf{Motivación.} Reduce superficie de error y esfuerzo de mantenimiento respecto de una solución propia.

\subsection{PostgreSQL (vía Supabase)}
Base de datos relacional donde reside el modelo de datos.
\begin{itemize}
    \item Tablas operativas: \textit{products}, \textit{categories}, \textit{sales}/\textit{sale\_items}, \textit{stock\_movements}.
    \item Datos analíticos y de contexto: \textit{demand\_analysis\_data}, \textit{predictions}/\textit{stored\_predictions}, \textit{weather\_data}, \textit{plans}, \textit{user\_subscriptions}.
    \item Soporte de integridad referencial, vistas e \textit{indexes} para consultas eficientes.
\end{itemize}
\noindent\textbf{Motivación.} Robustez y expresividad SQL para analítica operativa.

\subsection{PostgREST (Supabase)}
Capa que expone el esquema de PostgreSQL como API REST tipada.
\begin{itemize}
    \item La capa de servicios del frontend consume esta API para operaciones CRUD y consultas filtradas.
    \item Mantiene trazabilidad entre el modelo lógico y los endpoints, reduciendo código \textit{boilerplate}.
\end{itemize}
\noindent\textbf{Motivación.} Acelera la construcción de la capa de datos manteniendo convenciones REST.

\subsection{OpenWeatherMap API}
Fuente externa de variables meteorológicas usadas como factores exógenos.
\begin{itemize}
    \item Proporciona temperatura, estado del tiempo y otros atributos que se asocian a ventas y se persisten para análisis.
    \item Se contemplan \textit{fallbacks} ante indisponibilidad o latencia elevada.
\end{itemize}
\noindent\textbf{Motivación.} Cobertura suficiente para el ámbito del proyecto.

\subsection{Node.js y \texttt{npm}}
Entorno de ejecución y gestor de dependencias del proyecto.
\begin{itemize}
    \item Se emplea para el servidor de desarrollo, construcción del frontend y automatización de tareas (scripts).
    \item Estandariza \textit{linting}, empaquetado y gestión de versiones.
\end{itemize}
\noindent\textbf{Motivación.} \textit{Tooling} ampliamente adoptado en el ecosistema web.

\subsection{Utilitarios de CSV}
Conjunto de utilidades para ingesta histórica.
\begin{itemize}
    \item Validan formato, normalizan columnas y registran errores durante la carga de productos y ventas.
    \item Priorizan trazabilidad (filas aceptadas, rechazadas o corregidas) para reproducibilidad.
\end{itemize}
\noindent\textbf{Motivación.} Simplicidad y compatibilidad con exportaciones comunes.

\subsection{Buenas prácticas de configuración y seguridad}
\begin{itemize}
    \item Gestión de credenciales mediante variables de entorno (\texttt{.env}), evitando su versionado.
    \item Separación de entornos (desarrollo/producción) y rotación periódica de claves.
    \item En despliegues gestionados, aplicación de RLS por \textit{user\_id} para segmentación de datos.
\end{itemize}


\vspace{1cm}

\section{Diagrama de flujos}

Para describir el comportamiento dinámico del sistema incorporamos diagramas de flujo que muestran, paso a paso, cómo se procesan las operaciones clave. Estos esquemas complementan a los requerimientos y a la arquitectura estática, al hacer explícitas las secuencias, los puntos de decisión, los caminos alternativos y el manejo de errores, lo que facilita la validación funcional y el diseño de casos de prueba.
\vspace{1cm}


\subsection{DF-A · Autenticación}
El diagrama modela el control de acceso y el alta de usuarios. Se definió así para equilibrar seguridad (validaciones en el servicio de autenticación y verificación de correo), usabilidad (separación clara entre registro e inicio de sesión) y recuperabilidad (manejo de errores con retroalimentación inmediata). La convergencia final en el dashboard establece un único criterio de éxito y facilita la trazabilidad de estados; el objetivo es registrar las decisiones críticas del acceso sin describir pantallas intermedias.

\begin{figure}[!htbp]
  \centering
  \includegraphics[width=0.92\textwidth]{images/FlujoAutenticacion.jpg}
  \caption{DF-A · Autenticación -- Fuente: elaboración propia}
  \label{fig:df-a-autenticacion}
\end{figure}


\subsection{DF-B · Carga de ventas}
El diagrama describe el proceso de registro de ventas desde dos orígenes complementarios: ingreso manual tipo POS y carga automática (OCR/CSV). Se define así para cubrir los escenarios operativos más frecuentes con validaciones progresivas y una confirmación explícita antes del alta, incorporando puntos de corrección para minimizar errores y preservar trazabilidad. La bifurcación temprana optimiza tiempos según la fuente de datos y el cierre en confirmación establece una única condición de éxito.

\begin{figure}[!htbp]
  \centering
  \includegraphics[width=0.92\textwidth]{images/FlujoVentas.drawio.png}
  \caption{DF-B · Carga de ventas -- Fuente: elaboración propia}
  \label{fig:df-b-ventas}
\end{figure}


\subsection{DF-C · Planes y suscripciones}
El diagrama describe la gestión de planes dentro de la aplicación. Al crear una cuenta, el usuario recibe automáticamente el plan gratuito; por este motivo, cualquier cambio de plan se realiza exclusivamente desde la app. El flujo contempla verificación de identidad y estado de suscripción, selección del nuevo plan y confirmación (incluida la validación de pago si corresponde). Ante éxito se actualiza la suscripción y se reflejan las capacidades en sesión; ante error se conserva el plan previo. El enfoque prioriza control de acceso, trazabilidad y una transición segura entre planes.

\begin{figure}[!htbp]
  \centering
  \includegraphics[width=0.92\textwidth]{images/FlujoPlanes.drawio.png}
  \caption{DF-C · Planes y suscripciones -- Fuente: elaboración propia}
  \label{fig:df-c-planes}
\end{figure}


\vspace{1cm}
\section{Identidad de marca}\label{sec:brand}

La identidad de marca de \textbf{AIVA} se diseñó para comunicar tecnología confiable, simplicidad operativa y foco en pequeños comercios. Esta sección documenta los elementos rectores de esa identidad y su relación con la experiencia de uso del sistema.

\subsection{Naming}
\textbf{AIVA} es un acrónimo de \textit{Asistente Inteligente para Ventas y Análisis}. El nombre cumple tres criterios: (i) \textit{memorabilidad} y pronunciación sencilla en español e inglés, (ii) \textit{brevedad} apta para interfaces y piezas breves (íconos, botones, navegación), y (iii) \textit{connotación} directa con analítica y apoyo a la decisión. El nombre se usa en mayúsculas para reforzar la legibilidad y la consistencia visual en la interfaz.

\subsection{Misión}
Poner la analítica predictiva al alcance de micro y pequeñas empresas de alimentos perecederos, reduciendo mermas y quiebres mediante planificación basada en datos. La misión orienta decisiones de producto hacia soluciones de bajo costo, fáciles de adoptar y con impacto operativo medible.

\subsection{Visión}
Constituirse en la plataforma de referencia en América Latina para la gestión de demanda en comercios de proximidad, integrando múltiples fuentes de datos (histórico, clima, calendario) y promoviendo prácticas sustentables de producción y reposición.

\subsection{Paleta de colores}\label{subsec:paleta}

La paleta cromática de \textbf{AIVA} fue definida para comunicar confianza tecnológica, claridad y foco en la toma de decisiones. El eje visual está compuesto por un rango de azules–índigo que se aplica en fondos y navegación, desde un índigo profundo hasta un azul más luminoso (\#1E2A78–\#3F6BFF). Esta base fría evoca precisión y estabilidad, atributos propios de una plataforma analítica, y refuerza la percepción de fiabilidad por parte del usuario.

Como acento de marca se incorpora el violeta (\#7C3AED, con su variante clara \#A78BFA), que introduce una connotación de innovación y modernidad. Este matiz se reserva para elementos de énfasis —botones primarios, indicadores destacados y piezas de comunicación—, logrando contraste visual sin perder coherencia con el tono profesional del sistema. Complementariamente, el cian (\#38BDF8) funciona como color informativo en enlaces y ayudas contextuales, guiando la atención sin distraer del contenido principal.

Los colores semánticos se emplean para codificar estados del sistema y facilitar la lectura operativa: el verde (\#22C55E) comunica confirmaciones y resultados positivos, mientras que el naranja (\#F59E0B) señala advertencias y situaciones que requieren seguimiento. La selección de saturación y brillo busca que estos mensajes sean perceptibles de forma inmediata, manteniendo, al mismo tiempo, una estética sobria adecuada al ámbito empresarial.

La familia de neutros se utiliza para garantizar legibilidad tipográfica y jerarquía en las superficies: un gris muy oscuro para textos (\#0F172A) y un gris claro para fondos (\#F1F5F9), complementados por blanco (\#FFFFFF) en componentes de alta claridad. En consonancia con pautas de accesibilidad, se privilegian combinaciones de alto contraste en las vistas más frecuentes, asegurando que la interfaz conserve nitidez tanto en entornos luminosos como en configuraciones de bajo brillo.

En conjunto, el gradiente principal de índigo a azul, el acento violeta y los semánticos verde/naranja construyen un lenguaje visual consistente: profesional en su base, expresivo en sus acentos y funcional en la comunicación de estados. Esta identidad cromática sostiene la experiencia de uso de AIVA y refuerza su posicionamiento como asistente inteligente para la gestión de demanda.

\begin{figure}[!htbp]
  \centering
  \includegraphics[width=0.92\textwidth]{images/paleta-aiva.png}
  \caption{Paleta de Colores -- Fuente: elaboración propia}
  \label{fig:paleta-aiva}
\end{figure}

\subsection{Logo}
El signo marcario combina un \textbf{símbolo geométrico} (módulos rectangulares que sugieren el trazo de una “A” y la idea de bloques de datos) con el \textbf{logotipo} “AIVA” en tipografía \textit{sans serif} de alta legibilidad.

\begin{itemize}
    \item \textbf{Configuración principal (lockup).} Símbolo a la izquierda y logotipo a la derecha, alineados sobre la línea base. El conjunto se utiliza en positivo (blanco) sobre fondos indigo/azules o en negativo (indigo/violeta) sobre fondos claros.
    \item \textbf{Área de protección.} Mantener un margen libre alrededor del logo equivalente a la altura del símbolo para evitar interferencias con otros elementos.
    \item \textbf{Tamaños mínimos.} Asegurar que el ancho total no sea inferior a 24\,mm en impresos (o 160\,px en pantalla) para conservar la legibilidad de las formas internas.
    \item \textbf{Variantes.} 
    \begin{itemize}
        \item \textit{Monocromo}: uso en una sola tinta (blanco o negro) cuando las restricciones de producción lo requieran.
        \item \textit{Isotipo}: sólo el símbolo para favicons, íconos de app o espacios reducidos.
    \end{itemize}
    \item \textbf{Usos incorrectos.} No distorsionar, rotar ni aplicar efectos de sombra; no alterar la paleta definida; no combinar el logo con fondos de bajo contraste que comprometan la lectura.
\end{itemize}

\vspace{1cm}
\chapter{Análisis económico}\label{chapter04}

\section{Modelo de negocio}

El modelo de negocio de AIVA se orienta a ofrecer a comercios de alimentos perecederos una plataforma web que pronostica la demanda diaria por producto y habilita decisiones operativas basadas en datos. La propuesta enfatiza un costo total de propiedad acotado, tiempos de respuesta breves y explicabilidad de los resultados, de modo que cada recomendación pueda ser auditada y comprendida por el equipo del local. El enfoque prioriza el impacto económico directo mediante la reducción de mermas y quiebres de stock, la mejora de la rotación y un ahorro de tiempo en tareas de carga y análisis.

Con el objetivo de representar de manera estructurada la lógica de creación, entrega y captura de valor de la solución, se adopta como marco el \textit{Business Model Canvas} (BMC). Esta herramienta permite describir y analizar de forma sistemática los distintos componentes del modelo de negocio —propuesta de valor, clientes, canales, recursos, actividades, socios, estructura de costos e ingresos— y constituye la base conceptual que habilita posteriormente el desarrollo del análisis financiero. De este modo, el BMC actúa como puente entre la definición estratégica del proyecto y la evaluación cuantitativa de su viabilidad económica.

\subsection{Propuesta de Valor}
AIVA propone una plataforma web de fácil adopción para comercios que operan con productos perecederos (panaderías, confiterías y tiendas naturales), orientada a reducir mermas y faltantes mediante planificación basada en datos. La solución integra modelos de predicción de demanda con series temporales y variables exógenas (clima y feriados), un asistente conversacional para consultas operativas y un flujo de carga asistida por imágenes (OCR) que minimiza la fricción de ingreso de datos. El valor se manifiesta en una mejora de la disponibilidad de productos, una disminución del desperdicio y una optimización de los recursos de producción, traduciéndose en mayor eficiencia operativa y márgenes más saludables para el comerciante.

\subsection{Segmentos de Clientes}
El segmento objetivo está conformado por pequeñas y medianas empresas del sector de alimentos perecederos, con foco en panaderías y confiterías de barrio. Se trata de negocios con recursos limitados para soluciones corporativas complejas, pero con necesidad crítica de herramientas predictivas que acompañen decisiones diarias de producción y reposición.
En esta primera etapa de desarrollo, el proyecto se plantea como meta alcanzar un parque de mil usuarios activos, conformado por comercios de pequeña y mediana escala dentro del sector de alimentos perecederos. Esta cifra funciona como horizonte de referencia tanto para dimensionar la infraestructura tecnológica como para proyectar el análisis financiero del modelo de negocio.

\subsection{Canales}
La captación se apoya primordialmente en publicidad digital, para la cual se asigna el 50\% del valor de la suscripción (USD~15 por usuario activo). Este canal permite adquisición directa, medible y escalable. La estrategia se complementa con difusión orgánica (boca a boca) y presencia sectorial, a efectos de reforzar la marca en el ámbito de referencia del cliente objetivo.

\subsection{Relación con los Clientes}
El modelo de relación combina autoservicio digital para el registro y uso inicial con soporte técnico y funcional por correo electrónico. Este esquema preserva la escalabilidad sin desatender la asistencia necesaria durante la adopción y operación, especialmente en etapas de onboarding y calibración del modelo predictivo.

\subsection{Recursos Clave}
Los recursos críticos son la infraestructura y las capacidades analíticas. En backend, \emph{Supabase Pro} ofrece hasta 100{,}000 usuarios activos mensuales incluidos, 250~GB/mes de egress (con amplio margen frente al patrón de uso de 1{,}000 usuarios) y 8~GB de base de datos (con ampliación bajo demanda), además de 100~GB de almacenamiento. En frontend, \emph{Vercel Pro} proporciona despliegue continuo, edge network, caché y observabilidad, suficientes para garantizar disponibilidad y rendimiento frente a la meta de 1{,}000 usuarios. A ello se suman APIs de terceros: datos meteorológicos (OpenWeather), capacidades de reconocimiento óptico de caracteres (OCR) y modelos de lenguaje de gran escala provistos por OpenAI, que sustentan tanto el asistente conversacional como los procesos de interpretación de datos visuales. Estos recursos tecnológicos fortalecen la diferenciación funcional de AIVA y garantizan una experiencia de usuario fluida, explicable y de alto valor agregado.

\subsection{Actividades Clave}
Las actividades esenciales incluyen el desarrollo y mantenimiento de la plataforma, la mejora continua de los modelos de predicción (monitoreo de error y recalibración), la operación y observabilidad del servicio en producción, la ejecución de campañas de adquisición digital y la provisión de soporte al usuario final. La consistencia entre predicción y operación diaria del comercio es un objetivo permanente.

\subsection{Socios Clave}
Si bien la adquisición es directa, se consideran aliados sectoriales (cámaras de panaderos y confiterías) para facilitar el acceso al mercado objetivo y la validación funcional. En lo tecnológico, los proveedores de nube y APIs (Supabase, Vercel, OpenAI y clima) constituyen socios clave que habilitan escalabilidad, confiabilidad y rapidez de iteración.

\subsection{Estructura de Costos}
La estructura combina costos fijos y variables. Entre los fijos se contemplan las suscripciones de infraestructura (Supabase Pro y Vercel Pro) y las herramientas de operación. Entre los variables se incluyen publicidad digital (USD~15 por usuario activo al mes asignados a adquisición), uso incremental de APIs (LLM, OCR, clima), almacenamiento y egress por encima de los tramos incluidos, y el soporte operativo. Adicionalmente, cada alta entra en un período de prueba de un mes, durante el cual los costos recaen sobre AIVA sin ingresos asociados, por lo que representa un costo directo de adquisición. La inversión inicial (CapEx) en desarrollo e ingeniería se registra previo al lanzamiento del servicio.

\subsection{Fuentes de Ingresos}
El ingreso proviene de una suscripción mensual de USD~30 por usuario. Este flujo recurrente se modela con una tasa de cancelación (churn) mensual del 5\%, incorporando además la conversión del período de prueba (trial) al plan pago. La combinación de suscripción, retención y adquisición digital permite proyectar escalabilidad de ingresos acorde con el crecimiento del parque de clientes.

\subsection{Indicadores Financieros Clave}
Para evaluar la sostenibilidad económica del modelo se incorporan métricas estándar de negocios por suscripción.

\paragraph{Costo de Adquisición de Cliente (CAC).}
Sea $G$ el gasto de adquisición por alta y $\gamma$ la tasa de conversión de \emph{trial} a pago. El CAC se define como
\[
CAC \;=\; \frac{G}{\gamma}.
\]
Dado $G = \text{USD }15$ (50\% de la suscripción asignado a publicidad) y $\gamma = 0{,}42$, se obtiene
\[
CAC \;=\; \frac{15}{0{,}42} \approx \text{USD }35{,}7.
\]

\paragraph{Valor de Vida del Cliente (LTV).}
Con un ingreso medio por usuario (ARPU) igual al precio mensual del plan y $c$ como churn mensual, el LTV se aproxima como
\[
LTV \;=\; \frac{ARPU}{c}.
\]
Bajo $ARPU = \text{USD }30$ y $c = 0{,}05$, resulta
\[
LTV \;=\; \frac{30}{0{,}05} \;=\; \text{USD }600.
\]

\paragraph{Relación LTV/CAC.}
La razón $LTV/CAC$ sintetiza la eficiencia de adquisición:
\[
\frac{LTV}{CAC} \;=\; \frac{600}{35{,}7} \;\approx\; 16{,}8.
\]
En términos de buenas prácticas SaaS, valores superiores a 3 son deseables; el presente resultado indica un margen holgado para sostener inversión en marketing y acciones de retención sin comprometer la rentabilidad.



\section{Análisis financiero}

\subsection{Metodología}

El modelo financiero se estructura sobre un servicio por suscripción mensual de USD~30 con un mes de prueba inicial sin cargo. La inclusión de un período de prueba se fundamenta en evidencia empírica que demuestra cómo el diseño de los \textit{trials} puede incidir en la adopción y la conversión: estudios experimentales en plataformas SaaS confirman que extender o personalizar la duración de la prueba incrementa tanto su utilización como la probabilidad de que los usuarios se conviertan en clientes pagos (Zhang, 2025). En este marco, se adopta una tasa de conversión post‐trial del 42\% como supuesto de referencia, entendida como una hipótesis informada por la literatura, que posteriormente deberá ser contrastada con datos reales de operación.

En cuanto a la tasa de cancelación (\textit{churn}), los benchmarks de referencia para SaaS B2B muestran que, en servicios con un precio mensual entre USD~25 y USD~50, el churn promedio es de 7,3\% (Vitally, 2025). En este análisis se adopta un 6\% como supuesto operativo, lo que representa una postura conservadora que evita sobredimensionar la retención de usuarios.



La sostenibilidad del modelo se evalúa mediante métricas propias de negocios por suscripción: el costo de adquisición de cliente (CAC), el valor de vida del cliente (LTV) y la razón entre ambos indicadores. Estas métricas constituyen estándares de análisis en la literatura aplicada y en la práctica de evaluación de startups, ya que permiten determinar si los ingresos generados por cada cliente a lo largo de su permanencia superan ampliamente los costos de captación (Burkland Finance, 2024). A partir de estos supuestos, se construye un escenario base que constituye la referencia principal del análisis financiero. Sobre esta base, y modificando parámetros clave como la tasa de crecimiento, se elaboran los escenarios pesimista y optimista, lo que permite evaluar la sensibilidad del modelo frente a diferentes condiciones de mercado.



\subsection{Estructura de costos: CapEx}
La inversión inicial (\emph{CapEx}) corresponde principalmente a recursos humanos para el desarrollo del MVP y el despliegue productivo. Se contemplan seis roles, con tarifas medias y dedicación acumulada de seis meses (o equivalentes). El detalle se presenta en la Figura~\ref{fig:capex}.

\begin{figure}[!htbp]
  \centering
  \includegraphics[width=0.85\textwidth]{images/Capex.PNG}
  \caption{Detalle de CapEx por rol -- Fuente: elaboración propia}
  \label{fig:capex}
\end{figure}

\subsection{Estructura de costos: OpEx}
El \emph{OpEx} mensual combina componentes fijos y variables. Entre los fijos se incluyen hosting del frontend y base de datos/backend; entre los variables: publicidad proporcional a las altas, impuestos locales sobre ingresos brutos, comisiones de facturación/cobranza, uso de tokens de OpenAI y servicios de clima. La Figura~\ref{fig:costos-variables-fijas} resume los parámetros fijos empleados en el modelo.

\begin{figure}[!htbp]
  \centering
  \includegraphics[width=0.6\textwidth]{images/CostosVariablesFijas.PNG}
  \caption{Variables fijas para costos -- Fuente: elaboración propia}
  \label{fig:costos-variables-fijas}
\end{figure}

El detalle mensual del OpEx se presenta en las Figuras~\ref{fig:opex-anio1}, \ref{fig:opex-anio2} y \ref{fig:opex-anio3}, correspondientes a los tres primeros años de operación.

\begin{figure}[!htbp]
  \centering
  \includegraphics[width=0.92\textwidth]{images/OpexAño1.PNG}
  \caption{OpEx mensual, Año 1 -- Fuente: elaboración propia}
  \label{fig:opex-anio1}
\end{figure}

\begin{figure}[!htbp]
  \centering
  \includegraphics[width=0.92\textwidth]{images/OpexAño2.PNG}
  \caption{OpEx mensual, Año 2 -- Fuente: elaboración propia}
  \label{fig:opex-anio2}
\end{figure}

\begin{figure}[!htbp]
  \centering
  \includegraphics[width=0.92\textwidth]{images/OpexAño3.PNG}
  \caption{OpEx mensual, Año 3 -- Fuente: elaboración propia}
  \label{fig:opex-anio3}
\end{figure}


\subsection{Ingresos y dinámica de conversión}

En el escenario base del modelo financiero se adopta una tasa de crecimiento mensual de altas del 9\%. Este parámetro se encuentra alineado con los rangos reportados para startups de software como servicio (SaaS) en fases iniciales, donde la literatura especializada identifica tasas de expansión mensual de entre 5\% y 10\% como habituales en etapas de tracción temprana (Burkland Finance, 2024). Bajo este supuesto, los ingresos se reconocen exclusivamente por clientes pagos, es decir, aquellos que han atravesado el período de prueba y se han convertido efectivamente en suscriptores.

En cuanto a los costos variables asociados al uso de inteligencia artificial, el análisis estima un cargo aproximado de USD~0,35 por usuario activo mensual correspondiente a consultas a modelos de OpenAI (tanto para el asistente conversacional como para funciones de OCR). Este valor surge de proyectar el patrón de uso definido en el caso base (alrededor de 30 consultas diarias por usuario, con una media de tokens por interacción y las tarifas vigentes de la API), lo que permite dimensionar de manera realista el impacto unitario de la infraestructura cognitiva sobre el costo operativo.

Finalmente, los resultados consolidados permiten sintetizar los principales indicadores financieros del escenario base. En el horizonte de tres años, los ingresos acumulados alcanzan aproximadamente USD~929.270, mientras que los costos operativos totales (OPEX) ascienden a USD~313.841. La inversión inicial (CapEx), correspondiente al desarrollo de la plataforma y adquisición de equipamiento, se sitúa en USD~196.000 adicionales. Este resumen evidencia que, bajo los supuestos planteados, el modelo proyecta una senda de crecimiento sostenible en la que los ingresos superan ampliamente los costos, validando la viabilidad económica de la propuesta.


\begin{figure}[!htbp]
  \centering
  \includegraphics[width=0.92\textwidth]{images/CashflowAño1.PNG}
  \caption{Cashflow proyectado, Año 1 -- Fuente: elaboración propia}
  \label{fig:cashflow-anio1}
\end{figure}

\begin{figure}[!htbp]
  \centering
  \includegraphics[width=0.92\textwidth]{images/CashflowAño2.PNG}
  \caption{Cashflow proyectado, Año 2 -- Fuente: elaboración propia}
  \label{fig:cashflow-anio2}
\end{figure}

\begin{figure}[!htbp]
  \centering
  \includegraphics[width=0.92\textwidth]{images/CashflowAño3.PNG}
  \caption{Cashflow proyectado, Año 3 -- Fuente: elaboración propia}
  \label{fig:cashflow-anio3}
\end{figure}


\subsection{Flujo de fondos del caso base}

El flujo de fondos proyectado para el caso base integra los ingresos por suscripciones, los costos operativos (OPEX) y la inversión inicial en desarrollo y equipamiento (CapEx). La dinámica se construye sobre el supuesto de crecimiento mensual del 9\% en altas de usuarios, con una conversión del 42\% post--trial, una tasa de cancelación mensual del 6\% y una tasa anual del 20\% que se utiliza en los 3 escenarios propuestos. 

En el \textbf{Año 1}, los ingresos generados resultan todavía insuficientes para cubrir la inversión inicial y los costos operativos, lo que mantiene el flujo acumulado en valores negativos a lo largo de los primeros meses. A medida que avanza el año, la curva de ingresos se acelera y reduce progresivamente el déficit acumulado, pasando de --USD~196.000 en el inicio a --USD~170.845 al cierre del mes 12.  

Durante el \textbf{Año 2}, la progresión de clientes pagos permite un crecimiento sostenido en los ingresos. Este efecto acelera la reducción del déficit acumulado, que desciende desde --USD~164.618 en el mes 13 hasta --USD~25.475 al finalizar el mes 24. El negocio se aproxima al punto de equilibrio financiero (\textit{break-even}), el cual se alcanza en el transcurso del tercer año.  

Finalmente, en el \textbf{Año 3}, el flujo acumulado se torna positivo y refleja un proceso de recuperación de la inversión inicial. A partir del mes 27 se alcanza el \textit{break-even} (USD~45.744 acumulados), y desde allí los excedentes se amplían con rapidez hasta llegar a un saldo positivo de USD~419.427 al cierre del mes 36. Este comportamiento confirma la viabilidad económica del modelo bajo los parámetros establecidos en el escenario base.  

Las Figuras~\ref{fig:flujo-anio1}, \ref{fig:flujo-anio2} y \ref{fig:flujo-anio3} muestran en detalle la evolución mensual del flujo de fondos para cada uno de los tres años analizados.

\begin{figure}[!htbp]
  \centering
  \includegraphics[width=0.95\textwidth]{images/FlujodefondosAño1.PNG}
  \caption{Flujo de fondos proyectado -- Año 1. Fuente: elaboración propia}
  \label{fig:flujo-anio1}
\end{figure}

\begin{figure}[!htbp]
  \centering
  \includegraphics[width=0.95\textwidth]{images/FlujodefondosAño2.PNG}
  \caption{Flujo de fondos proyectado -- Año 2. Fuente: elaboración propia}
  \label{fig:flujo-anio2}
\end{figure}

\begin{figure}[!htbp]
  \centering
  \includegraphics[width=0.95\textwidth]{images/FlujodefondosAño3.PNG}
  \caption{Flujo de fondos proyectado -- Año 3. Fuente: elaboración propia}
  \label{fig:flujo-anio3}
\end{figure}



\subsection{Análisis comparativo de escenarios}

Con el objetivo de evaluar la robustez del modelo financiero se plantearon tres escenarios de crecimiento: pesimista (5\%), base (9\%) y optimista (12\%). Antes de examinar sus resultados, se definen brevemente los principales indicadores utilizados en la comparación:

\begin{itemize}
  \item \textbf{Valor Actual Neto (VAN)}: mide la diferencia entre los flujos de caja futuros descontados a una tasa determinada y la inversión inicial. Un VAN positivo indica que el proyecto genera valor económico sobre el costo de capital.
  \item \textbf{Tasa Interna de Retorno (TIR)}: representa la tasa de descuento que iguala a cero el VAN. Es decir, la rentabilidad implícita del proyecto. Cuanto mayor sea la TIR respecto al costo de capital, más atractivo es el emprendimiento.
  \item \textbf{Payback}: señala el tiempo requerido para recuperar la inversión inicial a través de los flujos netos de caja. A menor período de recuperación, menor es el riesgo financiero.
\end{itemize}

\paragraph{Escenario pesimista (5\% de crecimiento mensual).}  
En esta proyección el modelo alcanza ingresos acumulados por USD~463.716 al cabo de tres años, con un OPEX total de USD~140.543. El VAN resulta positivo, aunque reducido (USD~15.531), mientras que la TIR se ubica en 24\%. El período de repago de la inversión inicial se extiende a 2 años y 6 meses. Este escenario refleja que, aun con un ritmo bajo de crecimiento, la propuesta mantiene viabilidad económica, pero con márgenes ajustados y retornos moderados.

\paragraph{Escenario base (9\% de crecimiento mensual).}  
Se trata del caso central de análisis. Bajo estos parámetros los ingresos totales ascienden a USD~929.270 en tres años, con un OPEX acumulado de USD~313.842. El VAN se eleva a USD~191.940 y la TIR alcanza un 53\%. El payback se logra en 2 años y 2 meses. Este escenario confirma que, con una dinámica de crecimiento acorde a benchmarks de SaaS en etapas iniciales, el modelo no solo cubre con holgura la inversión inicial de USD~196.000, sino que además genera una rentabilidad significativa.

\paragraph{Escenario optimista (12\% de crecimiento mensual).}  
Bajo un desempeño más acelerado, los ingresos totales alcanzan USD~1.643.576 en el período analizado, con un OPEX proporcionalmente mayor (USD~605.135). El VAN asciende a USD~443.576 y la TIR alcanza el 75\%. El payback se produce en apenas 2 años. Este resultado muestra el potencial de escalabilidad del modelo: una mayor tracción comercial acelera la recuperación de la inversión y multiplica el valor económico generado.

\paragraph{Comparación y conclusiones.}  
El contraste entre los tres escenarios permite observar la sensibilidad del modelo a la tasa de crecimiento mensual de usuarios. Mientras que con un 5\% el proyecto mantiene apenas la viabilidad financiera, a partir del 9\% se consolida como rentable y sostenible, y con un 12\% despliega un perfil altamente atractivo para inversionistas. Esto confirma que la variable crítica del modelo no reside únicamente en los costos fijos o variables, sino en la capacidad de sostener un ritmo de crecimiento en la base de clientes que garantice retornos adecuados sobre la inversión.


\begin{figure}[!htbp]
  \centering
  \includegraphics[width=0.9\textwidth]{images/AnalisisFinal.PNG}
  \caption{Comparación de escenarios: pesimista, base y optimista. Fuente: elaboración propia}
  \label{fig:analisis-final}
\end{figure}
\chapter{Conclusi\'on}\label{conclussions}


El desarrollo del sistema AIVA permitió demostrar cómo la inteligencia artificial puede convertirse en una herramienta concreta al servicio de los pequeños y medianos comercios, ayudándolos a tomar decisiones más precisas, eficientes y sostenibles.  
AIVA no se concibió simplemente como un software, sino como un acompañante digital capaz de interpretar datos, anticipar necesidades y transformar información dispersa en conocimiento accionable. Su principal valor radica en su capacidad para traducir complejidad técnica en soluciones simples y de impacto directo sobre la realidad cotidiana de un negocio.

El sistema contribuye de manera tangible a la reducción del desperdicio de alimentos, uno de los problemas más significativos en el sector de productos perecederos. Al predecir con mayor exactitud la demanda futura, los comerciantes pueden planificar mejor sus compras y su producción diaria, evitando excesos que terminan en pérdidas económicas y en impactos ambientales innecesarios.  
De esta manera, AIVA no sólo optimiza la rentabilidad, sino que también fomenta un modelo de gestión más responsable y alineado con los principios de sostenibilidad y eficiencia energética.

Desde el punto de vista operativo, AIVA representa un salto de calidad en la toma de decisiones. El comerciante, que tradicionalmente dependía de la intuición o la experiencia acumulada, puede ahora apoyarse en datos concretos, visualizados en tiempo real y enriquecidos con información externa como el clima o la estacionalidad. Este enfoque transforma la administración del negocio en un proceso medible, predecible y controlable, liberando tiempo y reduciendo la incertidumbre.

El sistema también aporta en el plano humano. Su diseño simple, con una interfaz clara y un asistente conversacional intuitivo, democratiza el acceso a la tecnología, permitiendo que personas sin formación técnica puedan aprovechar las ventajas de la inteligencia artificial. En este sentido, AIVA actúa como un puente entre el mundo de los datos y la realidad cotidiana de los emprendedores, impulsando una verdadera inclusión digital en el ámbito comercial.

En síntesis, AIVA constituye una innovación con propósito: mejora la rentabilidad del comerciante, reduce el desperdicio de alimentos, promueve prácticas sostenibles y acerca la inteligencia artificial a quienes más pueden beneficiarse de ella.  
El proyecto deja en evidencia que la tecnología, cuando se diseña con empatía y orientación al valor, puede ser una aliada poderosa para construir negocios más inteligentes, eficientes y humanos.

\section{Resumen de aportes}

El desarrollo de \textbf{AIVA} constituye un aporte significativo tanto desde el punto de vista tecnológico como desde su impacto práctico en la gestión de comercios de productos perecederos. 
A lo largo del proyecto, se lograron avances concretos en cuatro dimensiones principales: optimización operativa, interacción inteligente mediante lenguaje natural, accesibilidad tecnológica y sostenibilidad económica y ambiental.

En primer lugar, AIVA demostró que es posible implementar un sistema de predicción de demanda preciso y accesible, basado en modelos de series temporales y variables exógenas. 
El sistema integra información histórica (ventas, precios, stock) con factores externos como clima, estacionalidad y feriados, generando pronósticos confiables que permiten planificar la producción diaria con mayor eficiencia. 
Este enfoque combina técnicas estadísticas (Holt-Winters, regresión lineal) y algoritmos de aprendizaje automático, incorporando métricas de evaluación (MAE, RMSE, MAPE) para garantizar la validez de los resultados. 
El resultado es una herramienta capaz de reemplazar métodos empíricos o basados únicamente en la intuición, por decisiones objetivas respaldadas por datos.

En segundo lugar, el proyecto introdujo un componente de interacción natural mediante un asistente conversacional potenciado por modelos de lenguaje extensos (LLM), lo que representa un avance notable en el campo de la analítica predictiva aplicada. 
El LLM actúa como un intermediario entre el usuario y la base de datos, capaz de interpretar preguntas formuladas en lenguaje natural (por ejemplo: ``¿cuántas ventas hubo esta semana?'' o ``¿qué producto tuvo mayor demanda en los días fríos?'') y traducirlas dinámicamente en consultas SQL parametrizadas, seguras y auditables. 

Este proceso se implementó a través de un motor \textbf{Text-to-SQL} desarrollado en el backend, que sigue un flujo estructurado:

\begin{itemize}
    \item \textbf{Clasificación de intención:} el modelo identifica qué tipo de información busca el usuario (ventas, predicciones, comparaciones, clima, etc.).
    \item \textbf{Generación de consulta SQL parametrizada:} el LLM produce una instrucción \texttt{SELECT} compatible con PostgreSQL, respetando una lista segura de tablas y columnas (\textit{allowlist}) y filtrando automáticamente por \texttt{user\_id}.
    \item \textbf{Ejecución controlada y recuperación de datos:} la aplicación ejecuta la consulta dentro de un entorno supervisado (Supabase) mediante herramientas internas que impiden operaciones no permitidas.
    \item \textbf{Síntesis y redacción del resultado:} el LLM interpreta los datos obtenidos y genera una respuesta narrativa en lenguaje natural, con formato legible, métricas clave y recomendaciones accionables.
\end{itemize}

El resultado es un asistente que combina análisis de datos, razonamiento contextual y comunicación natural, eliminando la necesidad de conocimientos técnicos y democratizando el acceso a la inteligencia artificial para pequeños comerciantes.

En tercer lugar, AIVA incorporó una arquitectura moderna, modular y segura, construida sobre tecnologías abiertas y escalables (Next.js, Supabase, OpenWeatherMap y OpenAI). 
El uso de reglas de seguridad a nivel de fila (\textit{Row-Level Security, RLS}) y la gestión centralizada de datos garantizan la privacidad y la integridad de la información de cada usuario, mientras que la modularidad del backend facilita la incorporación de nuevos modelos predictivos o fuentes de datos sin alterar la estructura general del sistema.

Por último, el proyecto se orienta hacia la sostenibilidad económica y ambiental. 
Al reducir el desperdicio de alimentos, optimizar niveles de stock y prevenir la sobreproducción, AIVA contribuye a mejorar la rentabilidad del negocio y a disminuir el impacto ambiental asociado a la gestión ineficiente de inventarios. 
De este modo, el sistema integra principios de economía circular y consumo responsable, alineándose con los Objetivos de Desarrollo Sostenible (ODS 12) de la ONU.

Desde el punto de vista académico y profesional, AIVA constituye una experiencia integral que combina ingeniería de software, bases de datos, aprendizaje automático, seguridad, diseño de interacción y responsabilidad social. 
El proyecto demuestra cómo la integración entre LLMs, analítica predictiva y usabilidad centrada en el usuario puede generar soluciones con impacto real, capaces de transformar la toma de decisiones en entornos comerciales cotidianos.

En síntesis, los aportes de AIVA pueden resumirse en una frase: 
convertir lenguaje en datos, datos en decisiones, decisiones en ahorro, y ahorro en sostenibilidad. 
Esa es la verdadera contribución de este trabajo al cruce entre la tecnología, la gestión y la innovación responsable.


\section{Trabajo futuro}

Si bien los resultados alcanzados con AIVA demuestran su valor y potencial, existen múltiples líneas de evolución que podrían fortalecer su impacto y ampliar su alcance. El trabajo futuro se orienta principalmente hacia la automatización total del flujo de datos, la mejora continua del modelo predictivo, la expansión funcional a nuevos entornos y la consolidación de AIVA como una plataforma regional de gestión inteligente.

\textbf{Integración con sistemas de gestión locales y POS:}  
Uno de los pasos más relevantes consiste en integrar AIVA con los principales sistemas de gestión y puntos de venta utilizados en Argentina, como \textit{Tango Gestión}, \textit{Contabilium}, \textit{Colppy}, \textit{Tiendanube} y \textit{MercadoShops}. Esta integración permitiría automatizar completamente la carga de ventas, el control de inventario y la actualización de precios, eliminando tareas manuales y reduciendo la posibilidad de errores humanos. De esta manera, el comerciante podría utilizar AIVA sin necesidad de modificar su sistema actual, acelerando la adopción tecnológica y garantizando una transición fluida hacia un entorno de gestión basado en datos. Además, esta interoperabilidad abriría la puerta a alianzas estratégicas con proveedores de software locales, consolidando a AIVA como un componente complementario dentro del ecosistema de gestión comercial.

\textbf{Extensión del asistente con carga de archivos e interpretación visual:}  
Durante las pruebas de usabilidad, la usuaria Milena González propuso incorporar la posibilidad de adjuntar imágenes o documentos directamente en el chat del asistente para que el sistema los interprete automáticamente.  
Esta funcionalidad permitiría que AIVA procese comprobantes, tickets o listas de productos mediante técnicas de visión artificial y reconocimiento óptico de caracteres (OCR), ampliando así las formas de interacción entre el usuario y el sistema.  
La integración de esta capacidad reforzaría el rol del asistente como un verdadero punto de ingreso de información, reduciendo el tiempo necesario para registrar datos y mejorando la eficiencia operativa de los comercios que manejan grandes volúmenes de ventas diarias.  


\textbf{Aprendizaje automático continuo (AutoML en producción):}  
Otra línea de trabajo clave es la incorporación de un módulo de aprendizaje automático continuo que permita que el sistema reentrene sus modelos de predicción de manera automática a medida que se registran nuevas ventas. Con un enfoque de \textit{AutoML en producción}, AIVA podría adaptarse dinámicamente a cambios en los patrones de consumo, la estacionalidad o las variaciones climáticas sin intervención del usuario. Este mecanismo garantizaría que el modelo mantenga un alto nivel de precisión en entornos cambiantes, generando predicciones cada vez más ajustadas a la realidad operativa de cada comercio. Asimismo, permitiría almacenar métricas de desempeño histórico, comparando la evolución de la precisión del modelo y fortaleciendo la confiabilidad del sistema a largo plazo.

\textbf{Aplicación móvil y alertas inteligentes:}  
La creación de una aplicación móvil nativa para Android e iOS representa otro paso natural en la evolución de AIVA. Esta aplicación permitiría que el comerciante acceda a métricas clave, reportes y predicciones desde cualquier lugar, así como recibir alertas inteligentes en tiempo real sobre situaciones críticas: bajo stock, exceso de inventario, productos con riesgo de vencimiento o condiciones meteorológicas que puedan alterar la demanda. Este componente potenciaría la agilidad operativa del usuario, facilitando la toma de decisiones rápidas y reduciendo el tiempo de reacción ante imprevistos. Además, reforzaría el carácter omnicanal de AIVA, alineándolo con las tendencias actuales de movilidad y disponibilidad permanente de la información.

\textbf{Expansión multisectorial y multilingüe:}  
Finalmente, se plantea la expansión de AIVA hacia nuevos sectores y mercados. El modelo puede adaptarse fácilmente a otros rubros de productos perecederos, como carnicerías, verdulerías, florerías o supermercados locales, ampliando así su base de usuarios potenciales. A su vez, la incorporación de capacidades multilingües permitiría escalar la plataforma a otros países de América Latina, ofreciendo una versión regionalizada del sistema bajo un modelo de software como servicio (\textit{SaaS}). Esta expansión convertiría a AIVA en una herramienta transversal para la digitalización del comercio minorista, promoviendo la adopción de la inteligencia artificial en contextos productivos reales y contribuyendo al desarrollo económico y tecnológico de la región.

En conjunto, estas líneas de trabajo representan la evolución natural de AIVA: de ser un asistente inteligente enfocado en la optimización de inventarios, a transformarse en una plataforma integral de gestión predictiva, móvil y conectada, capaz de acompañar al comerciante en todas las etapas de su operación. El futuro de AIVA se proyecta así como una síntesis entre tecnología, sostenibilidad y simplicidad, con el propósito de seguir reduciendo desperdicios, potenciando la rentabilidad y acercando la inteligencia artificial a quienes realmente la necesitan.






\printbibliography

\appendix
\input{chapters/appendix/annex}

\newpage
\listoffigures
\newpage
\listoftables

\end{document}
