\begin{Resumen} % \chapter*{Resumen}

    \indent El presente Proyecto Final de Ingeniería propone el desarrollo de un Sistema de análisis y predicción con inteligencia artificial para tiendas de productos perecederos. El proyecto tiene por objetivo optimizar la producción diaria de pequeños y medianos comercios de Buenos Aires en 2025 mediante la predicción de la demanda, reduciendo el desperdicio de alimentos y mejorando la rentabilidad.\\

    \indent La plataforma integrará modelos de series temporales e IA para generar pronósticos por artículo, procesará registros de venta extraídos de imágenes y ofrecerá un motor analítico que, a través de un dashboard y un chatbot conversacional, brindará indicadores clave y recomendaciones accionables al usuario. El MVP incluirá: módulo de predicción, visualización de KPIs, carga de imágenes, alertas de sobreproducción o faltantes y comparaciones semana a semana.\\

    \indent Entre los beneficios esperados destacan: reducción de mermas, incremento del margen operativo, profesionalización de la gestión y optimización de la compra de insumos, buscando dotar a los comercios perecederos de una herramienta innovadora que alinee su producción con la demanda real, contribuya a la sostenibilidad alimentaria y fortalezca su competitividad en un entorno económico desafiante.

\end{Resumen}
