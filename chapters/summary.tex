\begin{Resumen} % \chapter*{Resumen}

    \indent El presente Proyecto Final de Ingeniería propone el desarrollo de un Sistema de análisis y predicción con inteligencia artificial para tiendas de productos perecederos llamado AIVA. El proyecto tiene por objetivo optimizar la producción diaria de pequeños y medianos comercios de Buenos Aires en 2025\pdfcomment{Acá creo que sería mejor que reestructuren la oración, porque da a entender como que el proyecto solo va a servir para el 2025, y lo que están planteando sirve para mucho más. La idea es que el trabajo de ustedes no quede como una herramienta de estudio solo para el 2025 sino más bien como un MVP a largo plazo. Traten de dejar el 2025 porque creo que era una de las correcciones que les dieron (honestamente no lo recuerdo, revísenlo)} mediante la predicción de la demanda, reduciendo el desperdicio de alimentos y mejorando la rentabilidad.\\

    \indent \pdfcomment{Comentario sobre LaTeX. El comando indent solo hace falta en pocas ocasiones. Cuando ustedes dejan un salto de línea entre dos párrafos, LaTeX va a indentar por defecto el siguiente párrafo. Quizás les sirva conocer lo opuesto, que es que existe el comando nonindent, ya que les va a pasar que a veces se les van a indentar cosas que no quieran}La plataforma integrará modelos de series temporales e IA para generar pronósticos por artículo, procesará registros de venta extraídos de imágenes y ofrecerá un motor analítico que, a través de un dashboard y un chatbot conversacional, brindará indicadores clave y recomendaciones accionables al usuario. El MVP incluirá: módulo de predicción, visualización de KPIs, carga de imágenes, alertas de sobreproducción o faltantes y comparaciones semana a semana.\pdfcomment{Otro comentario más sobre LaTeX. Las doble barra como la que pusieron en LaTeX (ver el archivo tex en el que está este comentario) es para forzar un salto de línea cuando LaTeX no lo agrega. Acá es lo mismo. LaTeX lo hace por defecto, eso debería ir solo si no lo necesitan}\\

    \indent Entre los beneficios esperados destacan: reducción de mermas, incremento del margen operativo, profesionalización de la gestión y optimización de la compra de insumos, buscando dotar a los comercios perecederos de una herramienta innovadora que alinee su producción con la demanda real, contribuya a la sostenibilidad alimentaria y fortalezca su competitividad en un entorno económico desafiante. \pdfcomment{Si cambian algo en el resumen acuérdense de cambiarlo en el abstract}

\end{Resumen}
