\chapter*{Resumen}
\addcontentsline{toc}{chapter}{Resumen}
    El presente Proyecto Final de Ingeniería propone el desarrollo de un sistema de análisis y predicción basado en inteligencia artificial, denominado AIVA, orientado a comercios que comercializan productos perecederos. El objetivo principal es optimizar la producción diaria mediante la estimación precisa de la demanda, con el fin de reducir el desperdicio de alimentos y mejorar la rentabilidad del negocio. Para lograrlo, el sistema utiliza variables clave como el historial de ventas, el día de la semana, condiciones climáticas, fechas especiales y comportamiento de compra. Estas variables son fundamentales, ya que influyen directamente en los patrones de consumo. Al considerar estos elementos en el proceso de predicción, AIVA busca proporcionar información precisa y contextualizada que permita a los comerciantes tomar decisiones más informadas y eficientes.
    
    La plataforma integrará modelos de series temporales e IA para generar pronósticos por artículo, procesará registros de venta extraídos de imágenes y ofrecerá un motor analítico que, a través de un dashboard y un chatbot conversacional, brindará indicadores clave y recomendaciones accionables al usuario. El MVP incluirá: módulo de predicción, visualización de KPIs, carga de imágenes, alertas de sobreproducción o faltantes y comparaciones semana a semana.

    Entre los beneficios esperados destacan: reducción de mermas, incremento del margen operativo, profesionalización de la gestión y optimización de la compra de insumos, buscando dotar a los comercios perecederos de una herramienta innovadora que alinee su producción con la demanda real, contribuya a la sostenibilidad alimentaria y fortalezca su competitividad en un entorno económico desafiante. 


