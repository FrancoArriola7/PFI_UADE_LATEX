\chapter{Descripción}\label{chapter03}

AIVA es un sistema web orientado a pequeñas y medianas empresas de alimentos perecederos, como por ejemplo panaderías y confiterías, que busca reducir mermas y quiebres de stock mediante la planificación de producción y reposición basada en datos. La propuesta integra, en una única plataforma, la captura flexible de ventas (punto de venta, importación de CSV y carga asistida por imágenes), el enriquecimiento contextual (clima, calendario y patrones semanales) y un módulo de predicción de demanda diaria por producto. Sobre esta base, el sistema ofrece visualizaciones ejecutivas (backoffice), un asistente conversacional para consultas operativas y un mecanismo de alertas que anticipa situaciones de sobreproducción o faltantes. 

El núcleo analítico del sistema combina el historial transaccional con variables exógenas para estimar la demanda esperada a corto plazo. Este módulo implementa un enfoque heurístico transparente centrado en tendencia, estacionalidad y condiciones meteorológicas, con un diseño abierto a la incorporación de modelos estadísticos y de aprendizaje automático más avanzados. La arquitectura prioriza componentes de bajo costo y rápida adopción, asegurando trazabilidad de datos, reproducibilidad de resultados y una experiencia de usuario alineada con los flujos cotidianos del negocio.

\vspace{1cm}

\section{Requerimientos del sistema}
Esta sección especifica las condiciones verificables que delimitan el alcance de AIVA. Se formulan en modo imperativo (“El sistema debe …”) para facilitar su prueba y aceptación. Se organizan en dos grupos: funcionales (capacidades observables) y no funcionales (calidades y restricciones transversales).

\subsection{Requerimientos funcionales}
Definen lo que el sistema hace de cara al usuario y a los procesos de negocio (captura de datos, análisis y presentación). Recogen sólo lo esencial para operar AIVA extremo a extremo, sin imponer detalles de implementación.

\begin{enumerate}[label=RF-\arabic*., leftmargin=*, nosep]
  \item El sistema debe permitir registro, verificación de correo e inicio/cierre de sesión.
  \item El sistema debe asignar automáticamente un plan gratuito al alta y reflejarlo en la sesión.
  \item El sistema debe mantener la sesión activa y renovar credenciales mientras sean válidas.
  \item El sistema debe gestionar productos y categorías con operaciones CRUD.
  \item El sistema debe permitir cargar productos y ventas históricas vía CSV con validación previa.
  \item El sistema debe registrar ventas mediante un flujo POS con totales e ítems por producto.
  \item El sistema debe enriquecer transacciones con contexto (fecha, día, clima, feriados/fin de semana).
  \item El sistema debe generar predicciones de demanda por producto considerando tendencia, estacionalidad y clima.
  \item El sistema debe asociar a cada predicción nivel de confianza, factores explicativos y recomendación de stock.
  \item El sistema debe persistir predicciones por usuario y período y permitir su recalculo bajo demanda.
  \item El sistema debe exponer un tablero con KPIs de ventas, clima y predicciones vigentes.
  \item El sistema debe ofrecer un asistente conversacional para consultas y navegación guiada (\texttt{/api/chat}).
  \item El sistema debe generar y listar alertas operativas (stock bajo, sobreproducción potencial, feriados).
  \item El sistema debe filtrar todos los datos por el identificador del usuario autenticado.
\end{enumerate}

\subsection{Requerimientos no funcionales}
Establecen cualidades globales del sistema (seguridad, desempeño, mantenibilidad, confiabilidad). Limitan cómo se implementan los servicios, sin modificar el comportamiento funcional.

\begin{enumerate}[label=RNF-\arabic*., leftmargin=*, nosep]
  \item El sistema debe proteger recursos con autenticación/autorización y transmitir credenciales sólo por HTTPS.
  \item El sistema debe aplicar segmentación de datos en base (RLS o equivalente) por \texttt{user\_id}.
  \item El sistema debe ofrecer tiempos de respuesta percibidos menores a 2\,s en navegación del \textit{dashboard}.
  \item El sistema debe calcular predicciones bajo demanda en tiempos compatibles con uso interactivo (orden de segundos).
  \item El sistema debe manejar entradas inválidas y fallas externas con mensajes categorizados y recuperables.
  \item El sistema debe registrar eventos relevantes (errores, importaciones, cálculos) para auditoría.
  \item El sistema debe ser usable en navegadores modernos de escritorio y responder correctamente en distintos anchos.
  \item El sistema debe cumplir pautas básicas de accesibilidad (contraste, etiquetas/ARIA en controles principales).
  \item El sistema debe preservar integridad referencial y usar transacciones en altas de ventas.
  \item El sistema debe favorecer mantenibilidad con módulos tipados y convenciones de estilo consistentes.
  \item El sistema debe gestionar secretos vía variables de entorno y rotar credenciales periódicamente.
  \item El sistema debe respetar privacidad aplicable, realizar backups de datos críticos y versionar migraciones.
\end{enumerate}



\vspace{1cm}

\section{Interfaz Gráfica}

En esta sección se expone la interfaz gráfica desarrollada para la interacción cotidiana de los usuarios de AIVA en el entorno de comercio minorista de alimentos perecederos. La UI prioriza baja fricción operativa, consistencia visual y señales claras para la toma de decisiones, con patrones reutilizables que reducen la curva de aprendizaje. A continuación se describen las dos pantallas más relevantes por su impacto en el trabajo diario: el asistente conversacional (chatbot) y el módulo de predicción de demanda.

\subsection{Asistente conversacional (chatbot)}
\begin{figure}[!htbp]
  \centering
  \includegraphics[width=0.92\textwidth]{images/chatbotPage.png}
  \caption{Asistente conversacional (chatbot) -- Fuente: elaboración propia}
  \label{fig:ui-chatbot}
\end{figure}
La pantalla del asistente provee un canal de interacción en lenguaje natural para consultar métricas, navegar a vistas operativas y disparar acciones frecuentes sin abandonar el flujo de trabajo. El encabezado, con gradiente azul–violeta, muestra el estado del servicio mediante la insignia («En línea») y un selector de conversaciones que permite crear, cambiar y archivar hilos, favoreciendo la continuidad y la trazabilidad de consultas.

El área central es un historial desplazable con burbujas alineadas según el rol: los mensajes del usuario se muestran a la derecha y las respuestas del asistente, a la izquierda. Cada mensaje incorpora marca de tiempo y, en el caso del asistente, una acción para copiar el contenido. Cuando no hay historial, se presenta un saludo dinámico según la hora y atajos de acción (“Ver predicciones de mañana”, “Analizar ventas de esta semana”) que precargan consultas típicas para reducir fricción.

La barra inferior concentra la entrada de texto, con envío por Enter y salto de línea con \textit{Shift+Enter}, y botones para enviar o limpiar la conversación. Durante el procesamiento se indica actividad y se habilita la interrupción segura de la respuesta. La comunicación con el backend se realiza autenticada (token de sesión de Supabase) contra el endpoint interno \texttt{/api/chat}; se contemplan estados de error categorizados (cuota, servicio, autenticación, cancelación y red) con retroalimentación visual y mensajes recuperables. En conjunto, el diseño prioriza velocidad de uso, consistencia con el resto de la interfaz y control explícito sobre el ciclo de la conversación.

Cuando el usuario ingresa una consulta en el chat, el asistente procesa el mensaje y determina la intención de la pregunta a partir del lenguaje natural. En lugar de devolver una respuesta predefinida o aproximada, el agente genera de manera dinámica una consulta SQL parametrizada que traduce esa necesidad en una instrucción concreta sobre la base de datos. Esta consulta, siempre restringida a operaciones de solo lectura y filtrada por el identificador del usuario, se ejecuta en Supabase sobre el motor Postgres, garantizando seguridad y acceso únicamente a la información correspondiente. Una vez obtenidos los resultados, el asistente procesa los datos para calcular métricas, agrupar valores y elaborar indicadores relevantes, que luego son redactados en un formato comprensible y ejecutivo. De esta manera, el usuario recibe una respuesta clara y contextualizada, sustentada en datos reales y actualizados, sin exposición de detalles técnicos. El enfoque adoptado constituye el núcleo innovador de la solución: transformar de forma automática las consultas en lenguaje natural en consultas SQL seguras, ejecutarlas en tiempo real y presentar los resultados de manera amigable dentro del flujo conversacional.

\subsection{Predicción de demanda}
\begin{figure}[!htbp]
\centering
\includegraphics[width=0.95\textwidth]{images/prediccionPage.png}
\caption{Predicción de demanda -- Fuente: elaboración propia}
\label{fig:ui-prediccion}
\end{figure}
La pantalla de predicción concentra el cálculo y la consulta de estimaciones por producto. El encabezado permite elegir el período de análisis (hoy, semana, mes) y recalcular bajo demanda; al mismo tiempo, resume el estado del proceso con indicadores clave (ventas predichas, confianza alta, tendencia positiva y volumen de datos históricos). La sección analítica combina un gráfico de ventas reales filtrable por producto con tres vistas complementarias: un resumen ejecutivo (top productos y riesgos), un detalle tabular con filtros por producto/categoría y umbral de confianza, y un historial de ventas con contexto (clima, fin de semana, feriados). Las predicciones se generan en tiempo real mediante el servicio de predicción y se persisten en la base (tabla \texttt{predictions}, por usuario y período), priorizando trazabilidad, consistencia visual con el dashboard y manejo explícito de errores/estados para decisiones operativas rápidas.

Cuando se registra una venta en el sistema, además de almacenar la transacción, se capturan variables contextuales relevantes como fecha y hora, condiciones climáticas, feriados y características del producto (por ejemplo, si es perecedero). Estos datos se combinan con el historial reciente de ventas para estimar la demanda futura de cada producto, aplicando factores de tendencia, estacionalidad y contexto temporal. El resultado de este proceso es una predicción por período (hoy, semana o mes) que incluye métricas de cantidad esperada, nivel de confianza, tendencia y recomendaciones de stock. Las predicciones generadas se guardan automáticamente en la base de datos en la tabla \texttt{predictions}, asociadas al usuario y al período, lo que asegura que luego puedan ser consultadas en tiempo real desde la interfaz conversacional o visualizadas en la pantalla analítica. De este modo, cada nueva venta no solo actualiza el registro histórico, sino que alimenta y refina las estimaciones disponibles para la toma de decisiones.
\vspace{1cm}

\section{Tecnologías utilizadas}\label{sec:tecnologias}
A continuación se detallan las tecnologías empleadas en AIVA, explicando su rol dentro de la solución, el modo de integración y los criterios que justifican su elección. La selección prioriza un balance entre rapidez de implementación, bajo costo operativo, mantenibilidad y adecuación a los requerimientos del sistema.

\subsection{Next.js (React + TypeScript)}
Next.js es un \textit{framework} sobre React que incorpora enrutamiento, optimizaciones de rendimiento y opciones de renderizado (cliente/servidor). En AIVA se utiliza para:
\begin{itemize}
    \item Implementar la interfaz web con componentes reutilizables y tipado estático (TypeScript), reduciendo defectos en tiempo de compilación.
    \item Organizar el ruteo (App Router) separando vistas operativas (\textit{dashboard}, POS, predicción, reportes) de páginas auxiliares (autenticación, perfil).
    \item Favorecer una arquitectura mantenible (módulos, \textit{hooks} y servicios) con buena escalabilidad del frontend.
\end{itemize}
\noindent\textbf{Motivación.} Ecosistema maduro y documentación extensa, compatible con librerías modernas de UI.

\subsection{Tailwind CSS y \textit{shadcn/ui}}
Tailwind CSS provee utilidades de estilo de bajo nivel; \textit{shadcn/ui} aporta componentes accesibles basados en Tailwind.
\begin{itemize}
    \item Permiten maquetar formularios, tablas, modales y gráficos con consistencia visual y tiempos de desarrollo acotados.
    \item Estandarizan patrones de interacción (validaciones, estados de carga y error) y facilitan la personalización fina del diseño.
\end{itemize}
\noindent\textbf{Motivación.} Coherencia visual y productividad. 

\subsection{Recharts}
Recharts es una biblioteca de gráficos para React.
\begin{itemize}
    \item Se utiliza en el \textit{dashboard} y reportes para visualizar KPIs (ventas por período, productos destacados) y comparativas (predicción vs.\ observado).
    \item Ofrece componentes responsivos y composables, adecuados para vistas con filtros interactivos.
\end{itemize}
\noindent\textbf{Motivación.} Integración directa con React y bajo costo de aprendizaje.

\subsection{Supabase Auth}
Servicio gestionado para registro, inicio de sesión y manejo de sesiones (incluida verificación por correo).
\begin{itemize}
    \item Centraliza la identidad del usuario y emite \textit{tokens} para acceder a recursos protegidos.
    \item Simplifica la implementación de rutas seguras y del \textit{feature gating} por plan/suscripción.
\end{itemize}
\noindent\textbf{Motivación.} Reduce superficie de error y esfuerzo de mantenimiento respecto de una solución propia.

\subsection{PostgreSQL (vía Supabase)}
Base de datos relacional donde reside el modelo de datos.
\begin{itemize}
    \item Tablas operativas: \texttt{products}, \texttt{categories}, \texttt{sales}/\texttt{sale\_items}, \texttt{stock\_movements}.
    \item Datos analíticos y de contexto: \texttt{demand\_analysis\_data}, \texttt{predictions}/\texttt{stored\_predictions}, \texttt{weather\_data}, \texttt{plans}, \texttt{user\_subscriptions}.
    \item Soporte de integridad referencial, vistas e \textit{indexes} para consultas eficientes.
\end{itemize}
\noindent\textbf{Motivación.} Robustez y expresividad SQL para analítica operativa.

\subsection{PostgREST (Supabase)}
Capa que expone el esquema de PostgreSQL como API REST tipada.
\begin{itemize}
    \item La capa de servicios del frontend consume esta API para operaciones CRUD y consultas filtradas.
    \item Mantiene trazabilidad entre el modelo lógico y los endpoints, reduciendo código \textit{boilerplate}.
\end{itemize}
\noindent\textbf{Motivación.} Acelera la construcción de la capa de datos manteniendo convenciones REST.

\subsection{OpenWeatherMap API}
Fuente externa de variables meteorológicas usadas como factores exógenos.
\begin{itemize}
    \item Proporciona temperatura, estado del tiempo y otros atributos que se asocian a ventas y se persisten para análisis.
    \item Se contemplan \textit{fallbacks} ante indisponibilidad o latencia elevada.
\end{itemize}
\noindent\textbf{Motivación.} Cobertura suficiente para el ámbito del proyecto.

\subsection{Node.js y npm}
Entorno de ejecución y gestor de dependencias del proyecto.
\begin{itemize}
    \item Se emplea para el servidor de desarrollo, construcción del frontend y automatización de tareas (scripts).
    \item Estandariza \textit{linting}, empaquetado y gestión de versiones.
\end{itemize}
\noindent\textbf{Motivación.} \textit{Tooling} ampliamente adoptado en el ecosistema web.

\subsection{Utilitarios de CSV}
Conjunto de utilidades para ingesta histórica.
\begin{itemize}
    \item Validan formato, normalizan columnas y registran errores durante la carga de productos y ventas.
    \item Priorizan trazabilidad (filas aceptadas, rechazadas o corregidas) para reproducibilidad.
\end{itemize}
\noindent\textbf{Motivación.} Simplicidad y compatibilidad con exportaciones comunes.

\subsection{Buenas prácticas de configuración y seguridad}
\begin{itemize}
    \item Gestión de credenciales mediante variables de entorno (\texttt{.env}), evitando su versionado.
    \item Separación de entornos (desarrollo/producción) y rotación periódica de claves.
    \item En despliegues gestionados, aplicación de RLS por \textit{user\_id} para segmentación de datos.
\end{itemize}


\vspace{1cm}

\section{Arquitectura del sistema}

La arquitectura adopta un enfoque liviano de tres capas con servicios gestionados. La interfaz de usuario se implementa en \textit{Next.js}, que además expone una API interna (rutas \texttt{/api}) para articular lógica de negocio: validación, enriquecimiento con clima y llamadas al asistente. Los datos persisten en \textit{Supabase} (PostgreSQL) con autenticación gestionada y políticas RLS por \texttt{user\_id}. Dos servicios externos completan el contexto: \textit{OpenWeather} para variables meteorológicas y \textit{OpenAI} para el asistente conversacional. 

Las interacciones principales se agrupan en cuatro flujos:
\begin{enumerate}
    \item Autenticación y provisión de \textit{tokens} (JWT).
    \item Operaciones CRUD de productos y ventas.
    \item Generación y almacenamiento de predicciones (incluida trazabilidad de factores).
    \item Consultas conversacionales a través del endpoint \texttt{/api/chat}.
\end{enumerate}

\begin{figure}[!htbp]
  \centering
  \includegraphics[width=0.92\textwidth]{images/arquitecturaAIVA.png} % <-- reemplazar por tu archivo
  \caption{Arquitectura del sistema -- Fuente: elaboración propia. 
Los logotipos pertenecen a sus respectivas marcas.}
  \label{fig:arquitectura-aiva}
\end{figure}
\vspace{1cm}

\section{Diagrama de flujos}

Para describir el comportamiento dinámico del sistema incorporamos diagramas de flujo que muestran, paso a paso, cómo se procesan las operaciones clave. Estos esquemas complementan a los requerimientos y a la arquitectura estática, al hacer explícitas las secuencias, los puntos de decisión, los caminos alternativos y el manejo de errores, lo que facilita la validación funcional y el diseño de casos de prueba.
\vspace{1cm}


\subsection{DF-A · Autenticación}
El diagrama modela el control de acceso y el alta de usuarios. Se definió así para equilibrar seguridad (validaciones en el servicio de autenticación y verificación de correo), usabilidad (separación clara entre registro e inicio de sesión) y recuperabilidad (manejo de errores con retroalimentación inmediata). La convergencia final en el dashboard establece un único criterio de éxito y facilita la trazabilidad de estados; el objetivo es registrar las decisiones críticas del acceso sin describir pantallas intermedias.

\begin{figure}[!htbp]
  \centering
  \includegraphics[width=0.92\textwidth]{images/FlujoAutenticacion.jpg}
  \caption{DF-A · Autenticación -- Fuente: elaboración propia}
  \label{fig:df-a-autenticacion}
\end{figure}


\subsection{DF-B · Carga de ventas}
El diagrama describe el proceso de registro de ventas desde dos orígenes complementarios: ingreso manual tipo POS y carga automática (OCR/CSV). Se define así para cubrir los escenarios operativos más frecuentes con validaciones progresivas y una confirmación explícita antes del alta, incorporando puntos de corrección para minimizar errores y preservar trazabilidad. La bifurcación temprana optimiza tiempos según la fuente de datos y el cierre en confirmación establece una única condición de éxito.

\begin{figure}[!htbp]
  \centering
  \includegraphics[width=0.92\textwidth]{images/FlujoVentas.drawio.png}
  \caption{DF-B · Carga de ventas -- Fuente: elaboración propia}
  \label{fig:df-b-ventas}
\end{figure}


\subsection{DF-C · Planes y suscripciones}
El diagrama describe la gestión de planes dentro de la aplicación. Al crear una cuenta, el usuario recibe automáticamente el plan gratuito; por este motivo, cualquier cambio de plan se realiza exclusivamente desde la app. El flujo contempla verificación de identidad y estado de suscripción, selección del nuevo plan y confirmación (incluida la validación de pago si corresponde). Ante éxito se actualiza la suscripción y se reflejan las capacidades en sesión; ante error se conserva el plan previo. El enfoque prioriza control de acceso, trazabilidad y una transición segura entre planes.

\begin{figure}[!htbp]
  \centering
  \includegraphics[width=0.92\textwidth]{images/FlujoPlanes.drawio.png}
  \caption{DF-C · Planes y suscripciones -- Fuente: elaboración propia}
  \label{fig:df-c-planes}
\end{figure}


\vspace{1cm}
\section{Identidad de marca}\label{sec:brand}

La identidad de marca de \textbf{AIVA} se diseñó para comunicar tecnología confiable, simplicidad operativa y foco en pequeños comercios. Esta sección documenta los elementos rectores de esa identidad y su relación con la experiencia de uso del sistema.

\subsection{Naming}
\textbf{AIVA} es un acrónimo de \textit{Asistente Inteligente para Ventas y Análisis}. El nombre cumple tres criterios: (i) \textit{memorabilidad} y pronunciación sencilla en español e inglés, (ii) \textit{brevedad} apta para interfaces y piezas breves (íconos, botones, navegación), y (iii) \textit{connotación} directa con analítica y apoyo a la decisión. El nombre se usa en mayúsculas para reforzar la legibilidad y la consistencia visual en la interfaz.

\subsection{Misión}
Poner la analítica predictiva al alcance de micro y pequeñas empresas de alimentos perecederos, reduciendo mermas y quiebres mediante planificación basada en datos. La misión orienta decisiones de producto hacia soluciones de bajo costo, fáciles de adoptar y con impacto operativo medible.

\subsection{Visión}
Constituirse en la plataforma de referencia en América Latina para la gestión de demanda en comercios de proximidad, integrando múltiples fuentes de datos (histórico, clima, calendario) y promoviendo prácticas sustentables de producción y reposición.

\subsection{Paleta de colores}\label{subsec:paleta}

La paleta cromática de \textbf{AIVA} fue definida para comunicar confianza tecnológica, claridad y foco en la toma de decisiones. El eje visual está compuesto por un rango de azules–índigo que se aplica en fondos y navegación, desde un índigo profundo hasta un azul más luminoso (\#1E2A78–\#3F6BFF). Esta base fría evoca precisión y estabilidad, atributos propios de una plataforma analítica, y refuerza la percepción de fiabilidad por parte del usuario.

Como acento de marca se incorpora el violeta (\#7C3AED, con su variante clara \#A78BFA), que introduce una connotación de innovación y modernidad. Este matiz se reserva para elementos de énfasis —botones primarios, indicadores destacados y piezas de comunicación—, logrando contraste visual sin perder coherencia con el tono profesional del sistema. Complementariamente, el cian (\#38BDF8) funciona como color informativo en enlaces y ayudas contextuales, guiando la atención sin distraer del contenido principal.

Los colores semánticos se emplean para codificar estados del sistema y facilitar la lectura operativa: el verde (\#22C55E) comunica confirmaciones y resultados positivos, mientras que el naranja (\#F59E0B) señala advertencias y situaciones que requieren seguimiento. La selección de saturación y brillo busca que estos mensajes sean perceptibles de forma inmediata, manteniendo, al mismo tiempo, una estética sobria adecuada al ámbito empresarial.

La familia de neutros se utiliza para garantizar legibilidad tipográfica y jerarquía en las superficies: un gris muy oscuro para textos (\#0F172A) y un gris claro para fondos (\#F1F5F9), complementados por blanco (\#FFFFFF) en componentes de alta claridad. En consonancia con pautas de accesibilidad, se privilegian combinaciones de alto contraste en las vistas más frecuentes, asegurando que la interfaz conserve nitidez tanto en entornos luminosos como en configuraciones de bajo brillo.

En conjunto, el gradiente principal de índigo a azul, el acento violeta y los semánticos verde/naranja construyen un lenguaje visual consistente: profesional en su base, expresivo en sus acentos y funcional en la comunicación de estados. Esta identidad cromática sostiene la experiencia de uso de AIVA y refuerza su posicionamiento como asistente inteligente para la gestión de demanda.

\begin{figure}[!htbp]
  \centering
  \includegraphics[width=0.92\textwidth]{images/paleta-aiva.png}
  \caption{Paleta de Colores -- Fuente: elaboración propia}
  \label{fig:paleta-aiva}
\end{figure}

\subsection{Logo}
El signo marcario combina un \textbf{símbolo geométrico} (módulos rectangulares que sugieren el trazo de una “A” y la idea de bloques de datos) con el \textbf{logotipo} “AIVA” en tipografía \textit{sans serif} de alta legibilidad.

\begin{itemize}
    \item \textbf{Configuración principal (lockup).} Símbolo a la izquierda y logotipo a la derecha, alineados sobre la línea base. El conjunto se utiliza en positivo (blanco) sobre fondos índigo/azules o en negativo (índigo/violeta) sobre fondos claros.
    \item \textbf{Área de protección.} Se mantiene un margen libre alrededor del logo equivalente a la altura del símbolo para evitar interferencias con otros elementos.
    \item \textbf{Tamaños mínimos.} Se asegura que el ancho total no sea inferior a 24\,mm en impresos (o 160\,px en pantalla) para conservar la legibilidad de las formas internas.
    \item \textbf{Variantes.} 
    \begin{itemize}
        \item \textit{Monocromo}: uso en una sola tinta (blanco o negro) cuando las restricciones de producción lo requieran.
        \item \textit{Isotipo}: sólo el símbolo para favicons, íconos de app o espacios reducidos.
    \end{itemize}
    \item \textbf{Usos incorrectos.} No distorsionar, rotar ni aplicar efectos de sombra; no alterar la paleta definida; no combinar el logo con fondos de bajo contraste que comprometan la lectura.
\end{itemize}

\vspace{1cm}