\chapter{Descripción}\label{chapter03}

AIVA es un sistema web orientado a pequeñas y medianas empresas de alimentos perecederos, como por ejemplo panaderías y confiterías, que busca reducir mermas y quiebres de stock mediante la planificación de producción y reposición basada en datos. La propuesta integra, en una única plataforma, la captura flexible de ventas (punto de venta, importación de CSV y carga asistida por imágenes), el enriquecimiento contextual (clima, calendario y patrones semanales) y un módulo de predicción de demanda diaria por producto. Sobre esta base, el sistema ofrece visualizaciones ejecutivas (backoffice), un asistente conversacional para consultas operativas y un mecanismo de alertas que anticipa situaciones de sobreproducción o faltantes. 

El núcleo analítico del sistema combina el historial transaccional con variables exógenas para estimar la demanda esperada a corto plazo. Este módulo implementa un enfoque heurístico transparente centrado en tendencia, estacionalidad y condiciones meteorológicas, con un diseño abierto a la incorporación de modelos estadísticos y de aprendizaje automático más avanzados. La arquitectura prioriza componentes de bajo costo y rápida adopción, asegurando trazabilidad de datos, reproducibilidad de resultados y una experiencia de usuario alineada con los flujos cotidianos del negocio.

\vspace{1cm}

\section{Funcionalidades del sistema}

\begin{itemize}
    \item \textbf{Autenticación y perfiles de usuario}
    \begin{itemize}
        \item Registro e inicio de sesión mediante servicio gestionado (Supabase Auth).
        \item Persistencia de sesión y provisión de identidad a la aplicación.
        \item Gestión básica de perfil de usuario.
    \end{itemize}

    \item \textbf{Configuración y localización}
    \begin{itemize}
        \item Inicialización de la ubicación del usuario (por defecto, Ciudad Autónoma de Buenos Aires).
        \item Persistencia de la ubicación para consumo por servicios externos (clima).
    \end{itemize}

    \item \textbf{Gestión de productos y categorías}
    \begin{itemize}
        \item Crear, leer, actualizar y eliminar productos y categorías.
        \item Carga masiva de productos vía CSV con previsualización y validación básica.
        \item Ajustes manuales de stock y visualización de movimientos de stock.
    \end{itemize}

    \item \textbf{Registro y administración de ventas}
    \begin{itemize}
        \item Punto de Venta (POS): Flujo de carrito, totales y confirmación de venta. Alta de venta y detalle de ítems en base de datos. Enriquecimiento contextual de cada transacción para análisis de demanda.
        \item Importación de ventas históricas (CSV): Carga de archivos CSV con validaciones y normalización.
        \item Carga asistida por imágenes (OCR): Interfaz de carga y estados de procesamiento.
    \end{itemize}

    \item \textbf{Datos contextuales de clima}
    \begin{itemize}
        \item Integración con API meteorológica.
        \item Disponibilización de variables climáticas para análisis y predicción.
    \end{itemize}

    \item \textbf{Análisis de demanda}
    \begin{itemize}
        \item Consolidación de datos transaccionales enriquecidos.
        \item Consultas y visualizaciones exploratorias (tendencias por producto/categoría, comparativas temporales).
    \end{itemize}

    \item \textbf{Predicción de demanda}
    \begin{itemize}
        \item Generación de predicciones: Estimación diaria por producto mediante enfoque heurístico (tendencia, estacionalidad, día de semana y clima). Indicadores asociados a cada predicción.
        \item Persistencia y consulta de predicciones: Almacenamiento de predicciones para trazabilidad y auditoría. Gráficos comparativos entre predicción y ventas observadas.
        \item Predicciones resumidas para tableros: Consolidación de resultados para consumo eficiente por el dashboard.
    \end{itemize}

    \item \textbf{Backoffice (dashboard)}
    \begin{itemize}
        \item Indicadores ejecutivos (ventas, productos destacados, clima, predicciones vigentes).
        \item Visualizaciones de gráficos con filtros para seguimiento operativo.
    \end{itemize}

    \item \textbf{Reportes}
    \begin{itemize}
        \item Generación de reportes con agregados por período, producto y categoría.
        \item Gráficos de evolución y comparativas semana a semana.
    \end{itemize}

    \item \textbf{Alertas operativas}
    \begin{itemize}
        \item Listado de alertas de referencia (p. ej., stock bajo, sobreproducción potencial, feriados).
        \item Priorización de casos para acción correctiva.
    \end{itemize}

    \item \textbf{Asistente conversacional}
    \begin{itemize}
        \item Interfaz de chat para consultas frecuentes y navegación guiada.
        \item Respuestas basadas en reglas; integración con modelos de lenguaje planificada.
    \end{itemize}

    \item \textbf{Gestión de planes y suscripciones}
    \begin{itemize}
        \item Plan gratuito por defecto y soporte para planes pagos.
        \item Asociación de usuarios a planes y consulta del estado de suscripción.
    \end{itemize}

    \item \textbf{Perfil y preferencias}
    \begin{itemize}
        \item Edición de datos básicos de perfil.
        \item Configuraciones generales del usuario.
    \end{itemize}

    \item \textbf{Validaciones y manejo de errores}
    \begin{itemize}
        \item Validación de formularios y archivos (CSV).
        \item Mensajería de estados (carga, éxito, error) y prevención de acciones inválidas.
    \end{itemize}

    \item \textbf{Seguridad y segmentación de datos}
    \begin{itemize}
        \item Acceso autenticado a recursos y filtrado de datos por usuario.
        \item Preparado para políticas de control a nivel base de datos (RLS) en despliegues gestionados.
    \end{itemize}
\end{itemize}

\vspace{1cm}
\section{Diagrama de flujos}

Para describir el comportamiento dinámico del sistema incorporamos diagramas de flujo que muestran, paso a paso, cómo se procesan las operaciones clave. Estos esquemas complementan a los requerimientos y a la arquitectura estática, al hacer explícitas las secuencias, los puntos de decisión, los caminos alternativos y el manejo de errores, lo que facilita la validación funcional y el diseño de casos de prueba.

Diagramas a incluir:

DF-A · Autenticación

DF-B · Carga de venta

DF-C · Planes y suscripciones

