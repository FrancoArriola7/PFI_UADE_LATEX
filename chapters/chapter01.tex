\chapter{Introducción}

El desperdicio de alimentos constituye uno de los desafíos socio-ambientales más apremiantes de la actualidad. Estudios sobre retail alimentario local muestran, además, que por cada 100 pesos vendidos en categorías de frescos y almacén se pierden más de 3 pesos en mermas y obsolescencia, afectando márgenes ya de por sí estrechos \parencite{weteam2021}. Además, en la Argentina se pierde y desperdicia anualmente el 12,5\% de la producción agroalimentaria, equivalente a 16 millones de toneladas, de las cuales solo el 1,2\% corresponde a desperdicio en la etapa de comercialización, mientras que el resto se debe a pérdidas a lo largo de la cadena \parencite{tiscornia2022}. Esta situación implica un derroche de recursos naturales y económicos que afecta de manera directa la rentabilidad de los comercios que operan con productos frescos y perecederos.

El segmento de panaderías y pastelerías ilustra con crudeza esta problemática: en 2024 cerraron más de 400 locales, reflejando la dificultad de muchos pequeños y medianos negocios para ajustar su producción a una demanda cada vez más volátil (Pinto, 2024). 

Frente a este escenario, el presente Proyecto Final de Ingeniería propone el desarrollo de un sistema de análisis y predicción con inteligencia artificial para tiendas de productos perecederos, concebido como una plataforma web de fácil adopción. El sistema integrará modelos de series temporales y aprendizaje automático para pronosticar la demanda diaria a nivel de artículo, procesará registros de ventas capturados mediante carga de imágenes y empleará un motor analítico que —por medio de un dashboard interactivo y un chatbot conversacional— ofrecerá indicadores clave (KPIs) y recomendaciones operativas. El MVP contempla un módulo de predicción, visualización de métricas, ingestión inteligente de imágenes, alertas de sobreproducción o faltantes y comparaciones semana a semana.

Además de sus aportes técnicos, la propuesta busca contribuir al \emph{Objetivo de Desarrollo Sostenible 12} (Producción y Consumo Responsables) mediante la minimización de desperdicios y el uso eficiente de recursos. Al acercar tecnologías de IA a actores tradicionalmente rezagados en digitalización, se espera profesionalizar la gestión operativa, optimizar la compra de insumos y fortalecer la competitividad de cientos de unidades comerciales que conforman el tejido económico local.
