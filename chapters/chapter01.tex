\chapter{Introducción}

El desperdicio de alimentos constituye uno de los desafíos socio-ambientales más apremiantes de la actualidad. Estudios sobre retail alimentario local muestran, además, que por cada 100 pesos vendidos en categorías de frescos y almacén se pierden más de 3 pesos en mermas y obsolescencia, afectando márgenes ya de por sí estrechos \parencite{weteam2021}. Además, en la Argentina se pierde y desperdicia anualmente el 12,5\% de la producción agroalimentaria, equivalente a 16 millones de toneladas, de las cuales solo el 1,2\% corresponde a desperdicio en la etapa de comercialización, mientras que el resto se debe a pérdidas a lo largo de la cadena \parencite{tiscornia2022}. Esta situación implica un derroche de recursos naturales y económicos que afecta de manera directa la rentabilidad de los comercios que operan con productos frescos y perecederos.

El segmento de panaderías y pastelerías ilustra con crudeza esta problemática: en 2024 cerraron más de 400 locales, reflejando la dificultad de muchos pequeños y medianos negocios para ajustar su producción a una demanda cada vez más volátil (Pinto, 2024). 

Frente a este escenario, el presente Proyecto Final de Ingeniería propone el desarrollo de un sistema de análisis y predicción con inteligencia artificial para tiendas de productos perecederos, concebido como una plataforma web de fácil adopción. El sistema integrará modelos de series temporales y aprendizaje automático para pronosticar la demanda diaria a nivel de artículo, procesará registros de ventas capturados mediante carga de imágenes y empleará un motor analítico que —por medio de un dashboard interactivo y un chatbot conversacional— ofrecerá indicadores clave (KPIs) y recomendaciones operativas. El MVP contempla un módulo de predicción, visualización de métricas, ingestión inteligente de imágenes, alertas de sobreproducción o faltantes y comparaciones semana a semana.

Además de sus aportes técnicos, la propuesta busca contribuir al \emph{Objetivo de Desarrollo Sostenible 12} (Producción y Consumo Responsables) mediante la minimización de desperdicios y el uso eficiente de recursos. Al acercar tecnologías de IA a actores tradicionalmente rezagados en digitalización, se espera profesionalizar la gestión operativa, optimizar la compra de insumos y fortalecer la competitividad de cientos de unidades comerciales que conforman el tejido económico local.

\section{Objetivo General}

Desarrollar una plataforma web de predicción con inteligencia artificial para optimizar la producción diaria y reducir el desperdicio de productos perecederos en comercios, ajustando la oferta a la demanda real mediante datos históricos en el contexto de Buenos Aires 2025.\\

\noindent\textbf{Objetivos Específicos:}

\begin{itemize}
    \item Desarrollar un sistema de predicción de demanda diaria por producto utilizando inteligencia artificial, considerando variables como historial de ventas, día de la semana, clima y feriados.
    
    \item Implementar un módulo de carga y procesamiento automatizado de registros de venta, a partir de imágenes, mediante tecnologías de visión artificial y modelos de lenguaje.
    
    \item Diseñar una interfaz de backoffice con visualización de indicadores clave (KPIs), como márgenes, costos, productos más vendidos y ventas por franja horaria.
    
    \item Incorporar un asistente conversacional basado en lenguaje natural, que facilite la interacción del usuario con el sistema, permitiendo consultas operativas y recomendaciones automatizadas.
    
    \item Integrar un sistema de alertas y recomendaciones accionables, que anticipe sobreproducción o faltantes de stock y sugiera ajustes de planificación.
    
\end{itemize}

\section{Alcance}

El desarrollo de este producto de software será una aplicación web que incluirá las siguientes funcionalidades:

\begin{itemize}
    \item Se desarrollará un módulo de predicción de demanda diaria por producto, utilizando como variables de entrada el historial de ventas, día de la semana, clima y feriados, obteniendo como salida una predicción sobre la producción del producto. 
    \item Se incluirá un dashboard que permita visualizar los KPIs más relevantes para la organización, tales como márgenes de ganancia, costos de insumos, productos más vendidos y visualización de ventas por franja horaria. 
    \item Se implementará un chatbot que funcionará como interfaz principal de interacción con el sistema en el cual el usuario podrá realizar consultas.
    \item Se implementará un sistema de carga de imágenes por parte del usuario, permitiendo subir registros de ventas directamente desde la interfaz web, extrayendo y almacenando automáticamente los datos desde las imágenes mediante LLMs.
    \item Se implementará un conjunto de alertas básicas automatizadas, que advertirán al usuario en casos de potencial sobreproducción o faltantes de stock antelación, en base a la comparación entre los valores proyectados y los datos de producción ingresados. 
    \item Se incorporará una herramienta de comparación de ventas semana a semana, que facilitará la identificación de patrones de comportamiento del consumo y permitirá evaluar la evolución del desempeño comercial en el tiempo.
\end{itemize}
