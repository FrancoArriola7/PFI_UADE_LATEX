\chapter{Conclusi\'on}\label{conclussions}


El desarrollo del sistema AIVA permitió demostrar cómo la inteligencia artificial puede convertirse en una herramienta concreta al servicio de los pequeños y medianos comercios, ayudándolos a tomar decisiones más precisas, eficientes y sostenibles.  
AIVA no se concibió simplemente como un software, sino como un acompañante digital capaz de interpretar datos, anticipar necesidades y transformar información dispersa en conocimiento accionable. Su principal valor radica en su capacidad para traducir complejidad técnica en soluciones simples y de impacto directo sobre la realidad cotidiana de un negocio.

El sistema contribuye de manera tangible a la reducción del desperdicio de alimentos, uno de los problemas más significativos en el sector de productos perecederos. Al predecir con mayor exactitud la demanda futura, los comerciantes pueden planificar mejor sus compras y su producción diaria, evitando excesos que terminan en pérdidas económicas y en impactos ambientales innecesarios.  
De esta manera, AIVA no sólo optimiza la rentabilidad, sino que también fomenta un modelo de gestión más responsable y alineado con los principios de sostenibilidad y eficiencia energética.

Desde el punto de vista operativo, AIVA representa un salto de calidad en la toma de decisiones. El comerciante, que tradicionalmente dependía de la intuición o la experiencia acumulada, puede ahora apoyarse en datos concretos, visualizados en tiempo real y enriquecidos con información externa como el clima o la estacionalidad. Este enfoque transforma la administración del negocio en un proceso más científico, predecible y controlable, liberando tiempo y reduciendo la incertidumbre.

El sistema también aporta en el plano humano. Su diseño simple, con una interfaz clara y un asistente conversacional intuitivo, democratiza el acceso a la tecnología, permitiendo que personas sin formación técnica puedan aprovechar las ventajas de la inteligencia artificial. En este sentido, AIVA actúa como un puente entre el mundo de los datos y la realidad cotidiana de los emprendedores, impulsando una verdadera inclusión digital en el ámbito comercial.

En síntesis, AIVA constituye una innovación con propósito: mejora la rentabilidad del comerciante, reduce el desperdicio de alimentos, promueve prácticas sostenibles y acerca la inteligencia artificial a quienes más pueden beneficiarse de ella.  
El proyecto deja en evidencia que la tecnología, cuando se diseña con empatía y orientación al valor, puede ser una aliada poderosa para construir negocios más inteligentes, eficientes y humanos.

\subsection*{5.1 Resumen de aportes}

El desarrollo de AIVA constituye un aporte significativo tanto desde el punto de vista tecnológico como desde su impacto práctico en la gestión de comercios de productos perecederos.  
A lo largo del proyecto, se lograron avances concretos en tres dimensiones principales: la optimización operativa, la accesibilidad tecnológica y la sostenibilidad económica y ambiental.

En primer lugar, AIVA demostró que es posible implementar un sistema de predicción de demanda preciso y accesible, basado en inteligencia artificial y datos históricos, que permite a los comerciantes reducir pérdidas y planificar su producción con mayor eficiencia.  
El modelo integra variables internas (ventas, stock, precios) y externas (clima, estacionalidad), generando pronósticos confiables que se traducen en decisiones más racionales y objetivas. Este enfoque representa una mejora sustancial frente a los métodos tradicionales de estimación basados únicamente en la experiencia o la intuición.

En segundo lugar, el proyecto aportó una arquitectura moderna, modular y segura, construida sobre tecnologías abiertas y escalables (\textit{Next.js}, \textit{Supabase}, \textit{OpenWeatherMap} y \textit{OpenAI}).  
Esta combinación permitió desarrollar una solución robusta y adaptable, capaz de integrarse con otros sistemas y evolucionar con el tiempo, sin depender de infraestructura costosa ni de conocimientos técnicos avanzados.  
El uso de reglas de seguridad a nivel de fila (RLS) y la gestión centralizada de datos garantizan la privacidad y la integridad de la información de cada usuario.

Asimismo, AIVA incorporó una capa de interacción natural mediante un asistente conversacional, lo que representa un aporte innovador al enfoque de la analítica predictiva.  
Este componente humaniza la tecnología y facilita la adopción de la herramienta por parte de usuarios no especializados, democratizando el acceso a la inteligencia artificial en contextos comerciales de pequeña escala.

Otro aporte clave radica en la orientación hacia la sostenibilidad.  
Al reducir el desperdicio de alimentos, optimizar los niveles de stock y evitar sobreproducción, AIVA contribuye directamente a mejorar la rentabilidad del negocio y a disminuir el impacto ambiental asociado a la gestión ineficiente de inventarios.  
De este modo, el sistema integra los principios de la economía circular en la práctica cotidiana del comercio.

Finalmente, desde el punto de vista académico y profesional, el desarrollo de AIVA aportó una experiencia integral que combina la aplicación de conocimientos de ingeniería de software, bases de datos, aprendizaje automático y diseño centrado en el usuario.  
El proyecto demuestra cómo la unión entre rigor técnico y empatía por el usuario puede dar lugar a soluciones con impacto real en la sociedad.

En síntesis, los aportes de AIVA pueden resumirse en una frase: convertir datos en decisiones, decisiones en ahorro, y ahorro en sostenibilidad.
Esa es la verdadera contribución de este trabajo al cruce entre la tecnología, la gestión y la innovación responsable.


\section*{5.2 Trabajo futuro}

Si bien los resultados alcanzados con AIVA demuestran su valor y potencial, existen múltiples líneas de evolución que podrían fortalecer su impacto y ampliar su alcance. El trabajo futuro se orienta principalmente hacia la automatización total del flujo de datos, la mejora continua del modelo predictivo, la expansión funcional a nuevos entornos y la consolidación de AIVA como una plataforma regional de gestión inteligente.

\textbf{Integración con sistemas de gestión locales y POS:}  
Uno de los pasos más relevantes consiste en integrar AIVA con los principales sistemas de gestión y puntos de venta utilizados en Argentina, como \textit{Tango Gestión}, \textit{Contabilium}, \textit{Colppy}, \textit{Tiendanube} y \textit{MercadoShops}. Esta integración permitiría automatizar completamente la carga de ventas, el control de inventario y la actualización de precios, eliminando tareas manuales y reduciendo la posibilidad de errores humanos. De esta manera, el comerciante podría utilizar AIVA sin necesidad de modificar su sistema actual, acelerando la adopción tecnológica y garantizando una transición fluida hacia un entorno de gestión basado en datos. Además, esta interoperabilidad abriría la puerta a alianzas estratégicas con proveedores de software locales, consolidando a AIVA como un componente complementario dentro del ecosistema de gestión comercial.

\textbf{Aprendizaje automático continuo (AutoML en producción):}  
Otra línea de trabajo clave es la incorporación de un módulo de aprendizaje automático continuo que permita que el sistema reentrene sus modelos de predicción de manera automática a medida que se registran nuevas ventas. Con un enfoque de \textit{AutoML en producción}, AIVA podría adaptarse dinámicamente a cambios en los patrones de consumo, la estacionalidad o las variaciones climáticas sin intervención del usuario. Este mecanismo garantizaría que el modelo mantenga un alto nivel de precisión en entornos cambiantes, generando predicciones cada vez más ajustadas a la realidad operativa de cada comercio. Asimismo, permitiría almacenar métricas de desempeño histórico, comparando la evolución de la precisión del modelo y fortaleciendo la confiabilidad del sistema a largo plazo.

\textbf{Aplicación móvil y alertas inteligentes:}  
La creación de una aplicación móvil nativa para Android e iOS representa otro paso natural en la evolución de AIVA. Esta aplicación permitiría que el comerciante acceda a métricas clave, reportes y predicciones desde cualquier lugar, así como recibir alertas inteligentes en tiempo real sobre situaciones críticas: bajo stock, exceso de inventario, productos con riesgo de vencimiento o condiciones meteorológicas que puedan alterar la demanda. Este componente potenciaría la agilidad operativa del usuario, facilitando la toma de decisiones rápidas y reduciendo el tiempo de reacción ante imprevistos. Además, reforzaría el carácter omnicanal de AIVA, alineándolo con las tendencias actuales de movilidad y disponibilidad permanente de la información.

\textbf{Expansión multisectorial y multilingüe:}  
Finalmente, se plantea la expansión de AIVA hacia nuevos sectores y mercados. El modelo puede adaptarse fácilmente a otros rubros de productos perecederos, como carnicerías, verdulerías, florerías o supermercados locales, ampliando así su base de usuarios potenciales. A su vez, la incorporación de capacidades multilingües permitiría escalar la plataforma a otros países de América Latina, ofreciendo una versión regionalizada del sistema bajo un modelo de software como servicio (\textit{SaaS}). Esta expansión convertiría a AIVA en una herramienta transversal para la digitalización del comercio minorista, promoviendo la adopción de la inteligencia artificial en contextos productivos reales y contribuyendo al desarrollo económico y tecnológico de la región.

En conjunto, estas líneas de trabajo representan la evolución natural de AIVA: de ser un asistente inteligente enfocado en la optimización de inventarios, a transformarse en una plataforma integral de gestión predictiva, móvil y conectada, capaz de acompañar al comerciante en todas las etapas de su operación. El futuro de AIVA se proyecta así como una síntesis entre tecnología, sostenibilidad y simplicidad, con el propósito de seguir reduciendo desperdicios, potenciando la rentabilidad y acercando la inteligencia artificial a quienes realmente la necesitan.




