\anexo{Transcripción de entrevista a Milena de La Catalana}

\textbf{Entrevistador:} Para comenzar, ¿podrías contarnos tu nombre, el nombre del negocio y hace cuánto tiempo están trabajando?  

\textbf{Milena:} Mi nombre es Milena González. Trabajo en la confitería La Catalana, que es un negocio familiar que está en funcionamiento hace aproximadamente 15 años.  

\vspace{0.5em}

\textbf{Entrevistador:} ¿Cuántos empleados tiene su negocio?  

\textbf{Milena:} El negocio tiene 19 empleados.  

\vspace{0.5em}

\textbf{Entrevistador:} ¿Cuáles son los productos más y menos vendidos, y cuánto tiempo duran en promedio en el inventario?  

\textbf{Milena:} En general, los productos con más salida son las tortas, los sandwichitos de miga y las facturas.  
Los que menos se venden son los bombones y algunas masas finas.  
En promedio, los productos duran entre dos y tres días: el pan se elabora todos los días, mientras que las porciones y las masas tienen una duración de dos o tres días.  

\vspace{0.5em}

\textbf{Entrevistador:} ¿Cómo gestionan y estiman actualmente la cantidad de productos a producir o comprar cada día?  

\textbf{Milena:} Actualmente no usamos ningún sistema formal para gestionarlo. Estimamos la cantidad de productos a producir en base a nuestra experiencia, ya que cada día las ventas varían. Depende mucho si es fin de semana o no. Lo hacemos de forma intuitiva, basándonos en lo que sabemos que suele venderse más en cada día.  

\vspace{0.5em}

\textbf{Entrevistador:} ¿Tienen alguna forma de analizar las ventas pasadas para hacer ajustes en la producción? ¿Y qué nivel de importancia tiene para su negocio contar con una estimación precisa de la demanda?  

\textbf{Milena:} No contamos con ninguna forma de registrar o analizar las ventas pasadas. Para nosotros es muy importante contar con una estimación precisa de la demanda, ya que si calculamos mal, generamos una pérdida al negocio al no poder vender todos los productos elaborados.  

\vspace{0.5em}

\textbf{Entrevistador:} ¿Tienen algún sistema para predecir o estimar la demanda de productos?  

\textbf{Milena:} No contamos con ningún sistema para predecir la demanda, pero nos gustaría mucho adoptar alguno que nos permita hacer proyecciones. Sería muy útil para evitar el desperdicio y producir solo lo que realmente vamos a vender, o incluso para no quedarnos cortos con la producción cuando la demanda aumenta.  

\vspace{0.5em}

\textbf{Entrevistador:} ¿Estarían dispuestos a adoptar nuevas tecnologías si pudieran ayudar a reducir desperdicios y mejorar las ventas?  

\textbf{Milena:} La verdad que sí, nos interesaría mucho implementar alguna herramienta que nos pueda ayudar, siempre y cuando sea fácil e intuitiva de usar. Una solución así nos permitiría ahorrar muchos costos y optimizar el trabajo diario.  



\anexo{Transcripción de entrevista a Bruno de la confitería Salvador}

\textbf{Entrevistador:} Para comenzar, ¿podrías contarnos tu nombre, el nombre del negocio y hace cuánto tiempo está en funcionamiento?

\textbf{Bruno:} Mi nombre es Bruno y soy el propietario de la confitería Salvador. El negocio tiene más de 20 años de trayectoria.

\vspace{0.5em}

\textbf{Entrevistador:} ¿Cuántas personas trabajan actualmente en la confitería?

\textbf{Bruno:} Actualmente contamos con más de 10 empleados.

\vspace{0.5em}

\textbf{Entrevistador:} ¿Cuáles son los productos más vendidos en el local y cuáles tienen menor rotación?

\textbf{Bruno:} Lo que más se vende es el café y sus productos relacionados. En cambio, los postres suelen tener bastante menos salida.

\vspace{0.5em}

\textbf{Entrevistador:} ¿Cuánto tiempo permanecen en inventario los productos, en promedio?

\textbf{Bruno:} La rotación es diaria. Los productos duran aproximadamente un día.

\vspace{0.5em}

\textbf{Entrevistador:} ¿Cómo gestionan actualmente el inventario y la reposición de productos?

\textbf{Bruno:} Cada local se ocupa de su propia reposición. Es un proceso bastante manual.

\vspace{0.5em}

\textbf{Entrevistador:} ¿Y cómo estiman la cantidad de productos a producir o comprar cada día?

\textbf{Bruno:} Nos guiamos, sobre todo, por el faltante del día anterior. Es la referencia principal que usamos.

\vspace{0.5em}

\textbf{Entrevistador:} ¿Qué factores consideran al hacer estas estimaciones? Por ejemplo: ventas anteriores, clima, promociones, si es fin de semana, o ninguno.

\textbf{Bruno:} El factor más importante es si es fin de semana. Ahí siempre aumenta la demanda.

\vspace{0.5em}

\textbf{Entrevistador:} ¿Tienen alguna forma de registrar o analizar las ventas pasadas para ajustar la producción?

\textbf{Bruno:} Sí, contamos con registros de ventas. Los usamos para acomodar la producción, aunque de una manera bastante básica.

\vspace{0.5em}

\textbf{Entrevistador:} ¿Qué nivel de importancia tiene para su negocio contar con una estimación precisa de la demanda?

\textbf{Bruno:} Es muy importante. Es clave para evitar quedarnos sin stock o producir de más.

\vspace{0.5em}

\textbf{Entrevistador:} ¿Han experimentado pérdidas o desperdicios por sobreproducción o baja demanda?

\textbf{Bruno:} Sí, aunque no es mucho. Pero igual es algo que suma costos y tratamos de evitarlo.

\vspace{0.5em}

\textbf{Entrevistador:} ¿Utilizan actualmente alguna herramienta o software para gestionar inventario y ventas?

\textbf{Bruno:} Sí, usamos Maxirest. Sirve para registrar, pero no para anticipar la demanda ni para definir la producción del día.

\vspace{0.5em}

\textbf{Entrevistador:} ¿Estarían dispuestos a adoptar nuevas tecnologías si pudieran ayudar a reducir desperdicios y mejorar la eficiencia del negocio?

\textbf{Bruno:} Sí, claro. Todo lo que simplifique el trabajo y ayude a tomar decisiones más precisas es bienvenido.

\vspace{0.5em}

\textbf{Entrevistador:} Si existiera una herramienta capaz de predecir la demanda utilizando ventas pasadas, clima y otros factores externos, ¿le interesaría incorporarla?

\textbf{Bruno:} Sí, totalmente. Una herramienta así nos permitiría planificar con más seguridad. Poder anticipar la demanda nos ayudaría a ajustar la producción diaria, reducir mermas y organizarnos mejor. Sería muy útil para optimizar el negocio.
